\section{Begriffe}
	\subsection{Richtungsfeld}
	\begin{bem}
		Darstellung in der an jedem Punkt die Richtung der Tangente gezeichnet wird.
	\end{bem}
	\subsection{Allgemeine und Partikuläre Lösung}
	\begin{bem}
		Eine Lösung einer DGL $n$-ter Ordnung heißt allgemeine Lösung, wenn sie $n$ frei wählbare Parameter enthält, d.h. $y(x) = y(x,c_1,...,c_n)$. Eine allgemeine Lösung heißt vollständige Lösung, wenn Sie alle Lösungen enthält. Bei linearen DGL ist i.A. die allgemeine Lösung vollständig.
	\end{bem}
	