%%%%%Präambel%%%%%

\documentclass[12pt,a4paper]{report}%Schriftgröße, Papierformat einstellen
%\documentclass{scrbook}
\usepackage[top=30mm,bottom=30mm]{geometry}
\usepackage{lipsum}
\usepackage{csquotes}
%Pakete laden zur deutschen Rechtschreibung und für Umlaute
\usepackage[T1]{fontenc}
\usepackage[ngerman]{babel}
\usepackage[utf8]{inputenc} %für Windows, Linux
%\usepackage[applemac]{inputenc} %für Mac
%\usepackage{xcolor}
\usepackage[official]{eurosym}
\usepackage{graphicx}
\usepackage{caption}
\usepackage[dvipsnames]{xcolor}
\usepackage{cancel}
\usepackage{titlesec}
\usepackage{cite}
\usepackage{filecontents}
\usepackage{tabularx}
\usepackage{harvard}
\usepackage{units}
\usepackage{longtable} 
\usepackage{multirow}
\usepackage{chngcntr}
\usepackage{stmaryrd}
\usepackage{array}
\usepackage{autobreak}
\usepackage{booktabs}
\usepackage{float}
\usepackage{wrapfig}
\usepackage{hhline}
\let\harvardleftorig\harvardleft
%\usepackage[round]{natbib}
%\usepackage{hyperref}
\usepackage[nottoc,numbib]{tocbibind}
\usepackage{siunitx}
\usepackage{esvect}

%Pakete laden zu mathematischen Symbolen etc.
\usepackage{calc} 
\usepackage{amsmath,amssymb,amsthm,amsopn}
\usepackage{scrpage2}
\pagestyle{scrheadings}
\clearscrheadfoot
\automark[chapter]{section}
\ofoot{\pagemark}
\ifoot{}
\chead{\headmark}
\setfootsepline{1pt}
\setheadsepline{1pt}
%\setheadsepline[\textwidth+20pt]{0.5pt}

\DeclareMathOperator{\grad}{grad}
\DeclareMathOperator{\diverg}{div}
\DeclareMathOperator{\rot}{rot}
\DeclareMathOperator{\spur}{spur}
\DeclareMathOperator{\determ}{det}

%Inhaltsverzeichnis mit Links erstellen
\usepackage[colorlinks,
pdfpagelabels,
pdfstartview = FitH,
bookmarksopen = true,
bookmarksnumbered = true,
linkcolor = black,
plainpages = false,
hypertexnames = false,
citecolor = black] {hyperref}

% Umgebungen für Definitionen, Sätze, usw.
\newtheorem{satz}{Satz}[section]
\newtheorem{definition}[satz]{Definition}     
\newtheorem{lemma}[satz]{Lemma}	
\newtheorem{bem}{Bemerkung}[section]
% Es werden Sätze, Definitionen etc innerhalb einer Section mit
% 1.1, 1.2 etc durchnummeriert, ebenso die Gleichungen mit (1.1), (1.2) ..                  
\numberwithin{equation}{section}

\setcounter{secnumdepth}{4}
\setcounter{tocdepth}{4}

\titleformat{\paragraph}
{\normalfont\normalsize\bfseries}{\theparagraph}{1em}{}
\titlespacing*{\paragraph}
{0pt}{3.25ex plus 1ex minus .2ex}{1.5ex plus .2ex}

%neue Befehle definieren
\newcommand{\R}{\mathbb{R}} %zB \R als Abkürzung für das Symbol der reellen Zahlen
\newcommand{\N}{\mathbb{N}}
\newcommand{\Z}{\mathbb{Z}}
\newcommand{\Q}{\mathbb{Q}}
\newcommand{\C}{\mathbb{C}}
\newcommand{\diffp}{\partial}
\newcommand{\laplace}{\Delta}
%\newcommand{\diverg}{\operatorname{div}}

\newcommand{\subsubsubsection}{\paragraph}
\newcommand\citevgl
{\def\harvardleft{(vgl.\ \global\let\harvardleft\harvardleftorig}%
 \cite
}
\newcommand\citeVgl
{\def\harvardleft{(Vgl.\ \global\let\harvardleft\harvardleftorig}%
 \cite
}

\def\ezQu#1{'#1'}

\newcommand{\tabitem}{~~\llap{\textbullet}~~}

\def\ccite#1#2{\glqq #1\grqq\cite{#2}}

\newcolumntype{L}[1]{>{\raggedleft\let\newline\\\arraybackslash\hspace{0pt}}m{#1}}
%Makros
%Makro Color
%#1 Text
\def\colBord#1{\begingroup\color{Fuchsia}{#1}\endgroup}
\def\colRed#1{\begingroup\color{Red}{#1}\endgroup}
\def\colGreen#1{\begingroup\color{LimeGreen}{#1}\endgroup}
\def\colBlue#1{\begingroup\color{NavyBlue}{#1}\endgroup}

\def\usGreen#1#2{\underset{\colGreen{#1}}{#2}}
\def\usBord#1#2{\underset{\colBord{#1}}{#2}}

\def\ubGreen#1#2{\underbrace{#2}_{\colGreen{#1}}}

\def\defF{\textbf{Def.: }}
\def\mDef#1{
\begin{definition}
  #1
\end{definition}}

\def\vecT#1{\left(\begin{array}{c} #1 \end{array}\right)}
\def\dddot{\cdot \\ \cdot \\ \cdot}
\def\vecD#1{\vecT{#1_1 \\ \dddot \\ #1_d}}
\def\vecDt#1#2{\vecT{#1 \\ \dddot \\ #2}}
\def\vecN{\mathcal{O}}
\def\vspan#1{span \lbrace #1 \rbrace}
\def\vdim#1{dim \lbrace #1 \rbrace}
\def\vker#1{ker \lbrace #1 \rbrace}
\def\vrang#1{Rang \lbrace #1 \rbrace}
\def\mzxz#1#2#3#4{\left(\begin{array}{c c} #1 & #2 \\ #3 & #4 \\ \end{array}\right)}
\def\mdxd#1#2#3{\left(\begin{array}{c c c} #1 \\ #2 \\ #3 \end{array}\right)}
\def\dfp#1#2{\frac{\partial #1}{\partial #2}}
\def\diff#1#2{\frac{\mathrm{d}#1}{\mathrm{d}#2}}

\def\epsF{\pmb{\varepsilon}}

\def\multiTwo#1#2{\multicolumn{2}{>{\hsize=\dimexpr2\hsize+2\tabcolsep+\arrayrulewidth\relax}#1}{#2}}
\def\multiThree#1#2{\multicolumn{3}{>{\hsize=\dimexpr3\hsize+4\tabcolsep+2\arrayrulewidth\relax}#1}{#2}}

\def\inR#1{\qquad ,\; #1 \in \R}
\def\inRs{\in \R}
\def\bracks#1{\left[ #1 \right]}
\def\abs#1{\left| #1 \right|}
\def\brac#1{\left( #1 \right)}

%laziness
\def\fermi{Fermi-Dirac-Verteilung}
\def\ul#1{\underline{#1}}
\def\€{\euro{}}

\newcolumntype{L}[1]{>{\raggedleft\let\newline\\\arraybackslash\hspace{0pt}}m{#1}}
\newcolumntype{R}[1]{>{\raggedright\let\newline\\\arraybackslash\hspace{0pt}}m{#1}}
\newcolumntype{P}[1]{>{\centering\arraybackslash}p{#1}}



\def\formTab#1#2{
\begin{equation}
  \begin{tabularx}{12cm}{R{3cm} l l}
    #1 &: &$#2$
  \end{tabularx}
\end{equation}
}
\newcommand{\formTabL}[3]{
\begin{equation}
  \begin{tabularx}{12cm}{R{3cm} l l}
    #1 &: &$#2$ 
  \end{tabularx}
  \label{eq:#3}
\end{equation}}
\def\formTn{$ \\ $\;$ & $\;$ & $}
\def\formTnQ{$ \\ $\;$ & $\;$ & $\qquad}
\def\formTnQQ{$ \\ $\;$ & $\;$ & $\qquad \qquad}
\def\formTnQQQ{$ \\ $\;$ & $\;$ & $\qquad \qquad \qquad}

\renewcommand{\theequation}{\arabic{section}.\arabic{subsection}
.\arabic{equation}}
%Setzt den equation-Zaehler nach jeder Seite zurueck
%\numberwithin{equation}{subsection}	
\numberwithin{equation}{section}

%\setlength\abovedisplayskip{0pt}

% Auf der Seite http://detexify.kirelabs.org/classify.html können Sie mathematische Symbole, Pfeile usw per Maus eingeben und bekommen den Latex-Befehl dafür angezeigt.
% detexify gibt es auch als App...

%jetzt beginnt das eigentliche Dokument
\begin{document}
\bibliographystyle{agsm}

\author{}
\title{\underline{Elektrodynamik Zusammenfassung} \\ $\;$ \\ $\;$ \\ Florian Leuze}
\date{}
\maketitle % erzeugt den Kopf

$\;$ \newline
$\;$ \newline
$\;$ \newline
$\;$ \newline
$\;$ \newline
$\;$ \newline
$\;$ \newline
$\;$ \newline
$\;$ \newline
$\;$ \newline
$\;$ \newline
$\;$ \newline
\begin{table}[H]
  \centering
  \begin{tabular}{P{14cm}}
    \begin{LARGE}
		  \glqq Bring vor, was wahr ist; 
    \end{LARGE}\\
    \begin{LARGE}
		   Schreib' so, dass es klar ist
    \end{LARGE}\\
    \begin{LARGE}
		   Und verficht's, bis es mit dir gar ist!\grqq
    \end{LARGE}
  \end{tabular}
  \begin{tabular}{l}
    \begin{large}
      (Ludwig Boltzmann)
    \end{large}
  \end{tabular}
\end{table}

\newpage
\tableofcontents

\section*{Versionierung}
\begin{tabular}{|p{2cm}|p{1cm}|p{1.5cm}|p{8.5cm}|}\hline
Datum & Vers. & Kürzel & Änderung \\ \hline
27.11.2018 & 0.1 & FL & Erzeugung Dokument; Erzeugung Inhaltsverzeichnis; Erzeugung Versionierung; Erzeugung Literaturverzeichnis; 1.1-1.3\\ \hline
\end{tabular}
\listoffigures

 
\chapter{Elektrostatik} 
Die Elektrostatik beschreibt die Untersuchung von ruhenden Ladungsverteilungen. Zur Einordnung ist wichtig zu beachten, dass die Elektrostatik sich historisch gesehen vornehmlich mit dem makroskopischen beschäftigte. Daher sind Näherungen wie zu Beispiel \ezQu{Punktladung} oder \ezQu{elektrische Felder an einem Punkt} oft nur im makroskopischen zutreffend und im mikroskopischen irrelevant.

	\section{Coulombches Gesetz}
	Die  gesamte Elektrostatik beruht auf dem Coulombschen Gesetz. Es beschreibt die Kraft zwischen zwei geladenen Körpern, die sich relativ zueinander in Ruhe befinden. Coulomb zeigte experimentell für verhältnismäßig weit entfernte Ladungen, dass die Kraft zwischen ihnen :
	\begin{itemize}
		\item direkt proportional zum Betrag der Ladung ist,
		\item umgekehrt proportional zum Quadrat des relevanten Abstandes der Ladungen ist,
		\item in Richtung der Verbindungslinie zwischen den Ladungen wirkt,
		\item anziehend für Körper entgegengesetzter Ladung und abstoßend für Körper gleicher Ladung wirkt.
	\end{itemize}
	Ferner zeigte Coulomb auch, dass die Gesamtkraft mit der ein System von Ladungen auf einen sehr kleinen Körper wirkt gleich der Summe der einzelnen Zweikörperkräfte ist.
	
	\section{Elektrisches Feld}
		\subsection{Wichtige Einheiten}
		\renewcommand{\arraystretch}{1.5}
	\begin{longtable} {|p{2cm}|p{3cm}|p{8.4cm}|} \hline
	% Definition des Tabellenkopfes auf der ersten Seite
	%Spaltenbezeichnungen
	\textbf{Zeichen} & \textbf{Einheit} & \textbf{Bedeutung} \\
	\hline
	\endfirsthead % Erster Kopf zu Ende
	% Definition des Tabellenkopfes auf den folgenden Seiten
	%\caption{Abkürzungen/Formelzeichen}\\ \hline
	%Spaltenbezeichnungen
	%\textbf{Zeichen} & \textbf{Einheit} & \textbf{Bedeutung} \\
	%\hline
	%\endhead % Zweiter Kopf ist zu Ende
	%\multicolumn{3}{r}{Fortsetzung auf Folgeseite}\\
	\endfoot
	%\hline
	%\multicolumn{3}{r}{Ende} \\
	%\endlastfoot
	
	%a-g
	$Q$ & $C = As$ & Ladung \\ \hline
	$\vec{E}$ & $\frac{V}{m}$ & Elektrisches Feld \\ \hline
	%Sonderzeichen
	\end{longtable}
	\renewcommand{\arraystretch}{1}
		\subsection{Grundgleichungen und Stoffeigenschaften}
			\subsubsection{1. Gleichungsgruppe}
			\begin{equation}
				\rot{\vec{E}} = - \frac{\partial \vec{B}}{\partial t} \label{eq:mx_rot_e}
			\end{equation}
			\begin{equation}
				\rot \vec{H} = \vec{J}+\frac{\partial \vec{D}}{t} \label{eq:mx_rot_h}
			\end{equation}
			
			\subsubsection{2. Gleichungsgruppe}
			\begin{equation}
				\diverg \vec{D} = \varrho \label{eq:mx_div_d}
			\end{equation}
			\begin{equation}
				\diverg \vec{B} = 0 \label{eq:mx_div_b}
			\end{equation}
			
			\subsubsection{3. Gleichungsgruppe}
			\begin{equation}
				\vec{D} = \varepsilon \vec{E} \label{eq:mx_mat_d}
			\end{equation}
			\begin{equation}
				\vec{B} = \mu \vec{H}		\label{eq:mx_mat_b}		
			\end{equation}
			\begin{equation}
				\vec{J} = \kappa	\vec{E} \label{eq:mx_mat_j}
			\end{equation}
								
		\subsection{Elektrisches Feld}
		Das elektrische Feld ist die pro Ladungseinheit an einem gegebenen Punkt wirksame Kraft. Damit ist es eine vektorielle Funktion des Ortes und wird mit \textbf{E} bezeichnet. 
		\begin{equation}
			\vec{F} = q\vec{E}
		\end{equation}
		Wobei \textbf{F} die Kraft und $q$ die Ladung benennen. $q$ befindet sich an dem Punkt, an dem auch die Kraft gemessen wird. 
		In allgemeiner Form beschreibt das Coulombgesetz die Kraft zwischen zwei Ladungen an den Punkte $\textbf{x}_1$ und $\textbf{x}_2$,  wie folgt
		\begin{equation}
			\vec{F} = k q_2 q_1 \frac{\vec{x_2} - \vec{x_1}}{|\vec{x_2} - \vec{x_1}|^3} =  k \frac{q_1 q_2}{|\vec{r_{12}}|^2} \frac{\vec{r_{12}}}{|\vec{r_{12}}|}
		\end{equation}
		bzw. die das elektrische Feld an einem Punkt $\textbf{x}$ mit Abstand $\textbf{r}$ zur krafterzeugenden Ladung $Q$
		\begin{equation}
			\vec{E}(x) = k \frac{Q}{|\vec{r}|^2} \frac{\vec{r}}{|\vec{r}|}
		\end{equation}
		wobei $k$ in beiden Fällen eine vom Koordinatensystem abhängige Konstante ist. In elektrostatischen Einheiten (esE) wird beispielsweise $k=1$ gewählt. Im Si-System ist $k = (4 \pi  \varepsilon_0)^{-1}$.  \newline
		Wie bereits angesprochen lässt sich mit dem Coulombschen Gesetz eine Überlagerung mehrere Zweipunktkräfte die von einem Ladungssystem erzeugt sind beschreiben.
		\begin{equation}
			\vec{E}(x) = \frac{1}{4 \pi \varepsilon_0} \sum\limits_{i=1}^n q_i \frac{\vec{x} - \vec{x_i}}{|\vec{x} - \vec{x_i}|}
		\end{equation}
		Sind die Ladungen dabei so klein und zahlreich, dass sie durch eine Ladungsdichte $\varrho(\vec{\tilde{x})}$ beschrieben werden können lässt sich die Summe als Ladungs- bzw. Volumeninteral formulieren:
		\begin{equation}
			\vec{E}(x) = \frac{1}{4 \pi \varepsilon_0} \int \varrho(\vec{\tilde{x}})  \frac{\vec{x} - \vec{\tilde{x}}}{|\vec{x} - \vec{\tilde{x}}|} d(\tilde{x}, \tilde{y}, \tilde{z})
		\end{equation}
		
	\section{Der Gaußsche Integralsatz}
	Der Satz lautet in seiner allgemeinen Form ist durch
	\begin{equation}
			\int\limits_{\partial B}  \vec{v}\; \vec{d \sigma} = \int\limits_{\partial B} \left<\vec{v},\vec{n}\right> d \sigma = \int\limits_B \diverg \vec{v} \;d\vec{x}
	\end{equation}
	gegeben. Er ist auch unter der Bezeichnung Divergenztheorem in der Literatur zu finden. Wendet man ihn auf das elektrische Feld an erhält man  unter der Annahme kontinuierlicher Ladungsdichte und konstanter Dielektrizität:
	\begin{align}
		\diverg \vec{D} \overset{\eqref{eq:mx_mat_d}}{=} \diverg{\varepsilon E} \overset{\varepsilon = const.}{=} \diverg \vec{E} \nonumber\\
		\overset{\eqref{eq:mx_div_d}}{ \Rightarrow} \diverg \vec{E} = \varrho \text{ wenn } \varepsilon_0 \text{ konstant ist.} \nonumber \\
		\oint_B \left<\vec{E}, \vec{n}\right> \;d\sigma = \frac{1}{\varepsilon_0} \int\limits_V \varrho(\vec{x})\;d(x,y,z)
	\end{align}
	wobei $\varrho$ die Ladungsdichte angibt (hier konkret eine Flächenladungsdichte). $V$ ist dabei das von $B$ umschlossene Volumen. Analog gilt der Satz auch für die Newtonsche Gravitationskraft. Dort ersetzt man die Ladungsdichte durch die Massendichte.
	Setzt man nun erneut den Gaußschen Satz ein erhält man 
	\begin{align}
			\oint_B \left<\vec{E}, \vec{n}\right> \;d\sigma &= \int\limits_V \diverg \vec{v} \; d(x,y,z) \nonumber \\
			\Rightarrow \int\limits_V \diverg \vec{v} \; d(x,y,z) & = \frac{1}{\varepsilon_0} \int\limits_V \varrho(\vec{x})\;d(x,y,z) \nonumber \\
			\Rightarrow \int\limits_V \diverg \vec{v} \; d(x,y,z)  &-  \frac{1}{\varepsilon_0} \int\limits_V \varrho(\vec{x})\;d(x,y,z) = 0 \nonumber \\
			\Rightarrow \int\limits_V  \big( \diverg \vec{E} &- \frac{\varrho}{\varepsilon_0}  \big) \;d(x,y,z) = 0\\
	\end{align}
	Damit folgt jedoch sofort
	\begin{equation}
		\diverg \vec{E} = \frac{\varrho}{\varepsilon_0}
	\end{equation}
	Da sonst der Integrand nicht Null wäre. Diese Beziehung ist die differentielle Form des Gaußschen Gesetzes.
\newpage

\chapter{Anhänge}
\section{Hilfstabellen}
	\subsection{Abkürzungen/Formelzeichen} \label{ch:names}
	\renewcommand{\arraystretch}{1.5}
	\begin{longtable} {|p{2cm}|p{3cm}|p{8.4cm}|} \hline
	% Definition des Tabellenkopfes auf der ersten Seite
	%Spaltenbezeichnungen
	\textbf{Zeichen} & \textbf{Einheit} & \textbf{Bedeutung} \\
	\hline
	\endfirsthead % Erster Kopf zu Ende
	% Definition des Tabellenkopfes auf den folgenden Seiten
	\caption{Abkürzungen/Formelzeichen}\\ \hline
	%Spaltenbezeichnungen
	\textbf{Zeichen} & \textbf{Einheit} & \textbf{Bedeutung} \\
	\hline
	\endhead % Zweiter Kopf ist zu Ende
	\multicolumn{3}{r}{Fortsetzung auf Folgeseite}\\
	\endfoot
	\hline
	%\multicolumn{3}{r}{Ende} \\
	\endlastfoot
	
	%a-g
	$A$ & $m^2$ & Fläche \\ \hline
	$a$ & $\frac{m}{s^2}$ & Beschleunigung \\ \hline
	$b$ & $\frac{cm^2}{Vs}$ & Ladungsträgerbeweglichkeit \\ \hline
	$d$ & $m$ & Dicke \\ \hline
	$D_n$ & $\frac{m^2}{s}$ & Diffusionskonstante für Elektronen \\ \hline
	$D_p$ & $\frac{m^2}{s}$ & Diffusionskonstante für Löcher \\ \hline
	$e$ & $C$ & Elementarladung \\ \hline
	$E$ & $\frac{N}{C} = \frac{VAs}{mAs} = \frac{V}{m}$ & Elektrische Feldstärke \\ \hline
	$E_c$ & $eV$ & Leitungsbandkante \\ \hline
	$E_F$ & $eV$ & Fermi-Energie \\ \hline
	$E_g$ & $eV$ & Energie der Bandlücke \\ \hline
	$E_v$ & $eV$ & Valenzbandkante \\ \hline
	$f$ & $Hz$ & Frequenz \\ \hline
	$\vec{F}$ & $N = \frac{kgm}{s^2}$ & Kraft \\ \hline
	$G$ & $\frac{A}{V} = \frac{1}{\Omega} = S$ & Leitwert \\ \hline
	
	%h-n
	$h$ & $eVs$ & Planksches Wirkungsquantum\\ \hline
	$\hbar$ & $eVs$ & Dirac-Konstante \\ \hline
	$i$ & $A$ & Elektrischer Strom \\ \hline
	$j$ & $\frac{A}{m2}$ & Elektrische Stromdichte \\ \hline
	$J_n$ & $\frac{A}{m2}$ & Elektronenstromdichte \\ \hline
	$J_p$ & $\frac{A}{m2}$ & Löcherstromdichte \\ \hline
	$J_{diff}$ & $\frac{A}{m2}$ & Diffusionsstromdichte \\ \hline
	$J_{part}$ & $\frac{A}{m2}$ & Partikelstromdichte \\ \hline
	$J_to$ & $\frac{A}{m2}$ & Totale Stromdichte \\ \hline
	$J_r$ & $\frac{A}{m2}$ & Rekombinationsstromdichte \\ \hline
	$J_{drift}$ & $\frac{A}{m2}$ & Driftstromdichte \\ \hline
	$l$ & $m$ & Länge \\ \hline
	$L$ & $m$ & Minoritätsladungsträgerdiffusionslänge \\ \hline
	$L_n$ & $m$ & Diffusionslänge Elektronen \\ \hline
	$L_p$ & $m$ & Diffusionslänge Löcher \\ \hline
	
	%m-u
	$n$ & ... & Elektronenkonzentration \\ \hline
	$n_i$ & ... & Intrinsische Ladungsträgerdichte \\ \hline
	$n_{id}$ & ... & Idealität einer Diode \\ \hline
  $N_A$ & $m^{-3}$ & Akzeptorendichte \\ \hline
  $N_D$ & $m^{-3}$ & Donatorendichte \\ \hline
	$N_C$ & $cm^{-3}$ & Effektive Zustandsdichte der Elektronen \\ \hline
	$N_V$ & $cm^{-3}$ & Effektive Zustandsdichte der Löcher \\ \hline
	$p$ & ... & Lochkonzentration \\ \hline
	$q$ & $C$ & Probeladung (in der Regel = $e$) \\ \hline
	$\vec{r}$ & $m$ & Weg \\ \hline
	$r$ & $\Omega$ & Differentieller Widerstand \\ \hline
	$R$ & $\Omega$ & Widerstand \\ \hline
	$R_F$ & $\frac{\Omega}{square}$ & Flächenwiderstand \\ \hline 
	$U$ & $V$ & Elektrische Spannung \\ \hline
	$U_g$ & $V$ & Gesamtspannung \\ \hline
	 
	%v-z
	$v$ & $\frac{m}{s}$ & Geschwindigkeit \\ \hline
	$v_D, v_d$ & $\frac{m}{s}$ & Driftgeschwindigkeit \\ \hline
	$w$ & $m$ & Weite bzw. Breite  \\ \hline
	$W$ & $Ws = J = \frac{kgm^2}{s^2}$ & Arbeit bzw. Energie \\ \hline
	
	%griechisch
	$\alpha$ & $\frac{1}{^{\circ} C}$ & Temperturkoeffizient des Ohmwiderstandes \\ \hline
	$\nu$ & $Hz$ & Hier Frequenz der Welle \\ \hline
	$\rho$ & $\frac{V cm}{A} = \Omega  cm$ & Spezifischer Widerstand \\ \hline
	$\rho_e$ & ... & Ladungsdichte \\ \hline
	$\kappa$ & $\frac{1}{\Omega cm} = \frac{S}{cm}$ & Spezifische Leitfähigkeit \\ \hline
	$\varepsilon_0$ & $\frac{As}{Vm}$ & Dielektrizitätskonstante im Vakuum \\ \hline
	$\varphi$ & $V$ & Elektrisches Potential \\ \hline
	$\tau$ & $s$ & Stoßzeit \\ \hline
	$\tau$ & $s$ & Minoritätsladungsträgerlebensdauer \\ \hline
	$\mu$ & $\frac{cm^2}{Vs}$ & Beweglichkeit \\ \hline
	%Sonderzeichen
	\end{longtable}
	\renewcommand{\arraystretch}{1}
	
	\subsection{Wichtige Donatoren und Akzeptoren} \label{ch:don/acc}
	\renewcommand{\arraystretch}{1.5}
	\begin{longtable} {|p{2cm}|p{3cm}|p{8.4cm}|} \hline
	% Definition des Tabellenkopfes auf der ersten Seite
	%Spaltenbezeichnungen
	\textbf{Ch. Sym.} & \textbf{Name} & \textbf{Typ} \\
	\hline
	\endfirsthead % Erster Kopf zu Ende
	% Definition des Tabellenkopfes auf den folgenden Seiten
	\caption{Wichtige Donatoren und Akzeptoren}\\ \hline
	%Spaltenbezeichnungen
	\textbf{Zeichen} & \textbf{Einheit} & \textbf{Bedeutung} \\
	\hline
	\endhead % Zweiter Kopf ist zu Ende
	\multicolumn{3}{r}{Fortsetzung auf Folgeseite}\\
	\endfoot
	\hline
	%\multicolumn{3}{r}{Ende} \\
	\endlastfoot
	$B$ & Bor & Akzeptor \\ \hline
	$Al$ & Alluminium & Akzeptor \\ \hline
	$Ga$ & Gallium & Akzeptor \\ \hline
	$In$ & Indium & Akzeptor \\ \hline
	$P$ & Phosphor & Donator \\ \hline
	$As$ & Arsen & Donator \\ \hline
	$Sb$ & Antimon & Donator \\ \hline
	$Bi$ & Wismut & Donator \\ \hline
	\end{longtable}
	\renewcommand{\arraystretch}{1}
	\newpage
	\subsection{Effektive Massen} \label{ch:don/acc}
	\renewcommand{\arraystretch}{1.5}
	\begin{longtable} {|p{2cm}|p{3cm}|p{8.4cm}|} \hline
	% Definition des Tabellenkopfes auf der ersten Seite
	%Spaltenbezeichnungen
	\textbf{Band} & \textbf{Wert} & \textbf{Element} \\
	\hline
	\endfirsthead % Erster Kopf zu Ende
	% Definition des Tabellenkopfes auf den folgenden Seiten
	\caption{Effektive Massen}\\ \hline
	%Spaltenbezeichnungen
	\textbf{Band} & \textbf{Wert} & \textbf{Element} \\
	\hline
	\endhead % Zweiter Kopf ist zu Ende
	\multicolumn{3}{r}{Fortsetzung auf Folgeseite}\\
	\endfoot
	\hline
	%\multicolumn{3}{r}{Ende} \\
	\endlastfoot
	$\frac{m_n^*}{m_0}$ & $1,08$ & Silizium \\ \hline
	$\frac{m_n^*}{m_0}$ & $1,561$ & Germanium \\ \hline
	$\frac{m_n^*}{m_0}$ & $1,067$ & Gallium-Arsenid \\ \hline
	$\frac{m_p^*}{m_0}$ & $1,10$ & Silizium \\ \hline
	$\frac{m_p^*}{m_0}$ & $1,291$ & Germanium \\ \hline
	$\frac{m_p^*}{m_0}$ & $1,473$ & Gallium \\ \hline
	\end{longtable}
	\renewcommand{\arraystretch}{1}
	
	\subsection{Bandlücken wichtiger Materialien} \label{ch:gaps}
	\renewcommand{\arraystretch}{1.5}
	\begin{longtable} {|p{2cm}|p{3cm}|p{8.4cm}|} \hline
	% Definition des Tabellenkopfes auf der ersten Seite
	%Spaltenbezeichnungen
	\textbf{Zeichen} & \textbf{Wert in \text{e}V} & \textbf{Material} \\
	\hline
	\endfirsthead % Erster Kopf zu Ende
	% Definition des Tabellenkopfes auf den folgenden Seiten
	\caption{Bandlücken wichtiger Materialien}\\ \hline
	%Spaltenbezeichnungen
	\textbf{Zeichen} & \textbf{Einheit} & \textbf{Bedeutung} \\
	\hline
	\endhead % Zweiter Kopf ist zu Ende
	\multicolumn{3}{r}{Fortsetzung auf Folgeseite}\\
	\endfoot
	\hline
	%\multicolumn{3}{r}{Ende} \\
	\endlastfoot
	$E_{g,SiO_2}$ & $9$ & Siliziumdioxid \\ \hline
	$E_{g,C}$ & $5,47$ & Diamant \\ \hline
	$E_{g,CdS}$ & $2,42$ & Cadmiumsulfid \\ \hline
	$E_{g,GaP}$ & $2,26$ & Galliumphosphid \\ \hline
	$E_{g,GaAs}$ & $1,42$ & Gallium-Arsenid \\ \hline
	$E_{g,InP}$ & $1,35$ & Indiumphosphid \\ \hline
	$E_{g,Si}$ & $1,12$ & Silizium \\ \hline
	$E_{g,Ge}$ & $0,66$ & Germanium \\ \hline
	$E_{g,InSb}$ & $0,17$ & Indiumantimonid \\ \hline
	\end{longtable}
	\renewcommand{\arraystretch}{1}
	
	\subsection{Eckdaten wichtiger Halbleiter} \label{ch:eckd}
	\renewcommand{\arraystretch}{1.5}
	\begin{longtable} {|p{2cm}|p{2.6cm}|p{2.6cm}|p{2.6cm}|p{2.7cm}|} \hline
	% Definition des Tabellenkopfes auf der ersten Seite
	%Spaltenbezeichnungen
	\textbf{Ch. Sym.} & \textbf{$E_g$ in $\bracks{eV}$} & \textbf{$N_C$ in $\bracks{cm^{-3}}$} & \textbf{$N_V$ in $\bracks{cm^{-3}}$} & \textbf{$n_i$ in $\bracks{cm^{-3}}$} \\
	\hline
	\endfirsthead % Erster Kopf zu Ende
	% Definition des Tabellenkopfes auf den folgenden Seiten
	\caption{Eckdaten wichtiger Halbleiter}\\ \hline
	%Spaltenbezeichnungen
	\textbf{Ch. Sym.} & \textbf{$E_g$ in $\bracks{eV}$} & \textbf{$E_g$ in $\bracks{eV}$} & \textbf{$E_g$ in $\bracks{eV}$} & \textbf{$E_g$ in $\bracks{eV}$} \\
	\hline
	\endhead % Zweiter Kopf ist zu Ende
	\multicolumn{3}{r}{Fortsetzung auf Folgeseite}\\
	\endfoot
	\hline
	%\multicolumn{3}{r}{Ende} \\
	\endlastfoot
	Si & $1,124$ & $2,81 \cdot 10^{19}$ & $2,88 \cdot 10^{19}$ & $1,04 \cdot 10^{10}$ \\ \hline
	Ge & $0,67$ & $1,05 \cdot 10^{19}$ & $3,92 \cdot 10^{18}$ & $1,55 \cdot 10^{13}$ \\ \hline
	GaAs & $1,424$ & $4,33 \cdot 10^{17}$ & $8,13 \cdot 10^{18}$ & $2,04 \cdot 10^{6}$ \\ \hline
	\end{longtable}
	\renewcommand{\arraystretch}{1}
	\newpage
\subsection{Niederfeld- und Niederdotierungsbeweglichkeiten ($T = 300K$)} \label{ch:bewegl.}
	\renewcommand{\arraystretch}{1.5}
	\begin{longtable} {|p{2.4cm}|p{3.5cm}|p{3.5cm}|p{3.5cm}|} \hline
	% Definition des Tabellenkopfes auf der ersten Seite
	%Spaltenbezeichnungen
	\textbf{$n/p$} & \textbf{Si} & \textbf{Ge} & \textbf{GaAs} \\
	\hline
	\endfirsthead % Erster Kopf zu Ende
	% Definition des Tabellenkopfes auf den folgenden Seiten
	\caption{Niederfeld- und Niederdotierungsbeweglichkeiten}\\ \hline
	%Spaltenbezeichnungen
	\textbf{$n/p$} & \textbf{Si} & \textbf{Ge} & \textbf{GaAs} \\
	\hline
	\endhead % Zweiter Kopf ist zu Ende
	\multicolumn{3}{r}{Fortsetzung auf Folgeseite}\\
	\endfoot
	\hline
	%\multicolumn{3}{r}{Ende} \\
	\endlastfoot
	  $\mu_n \bracks{\frac{cm^2}{Vs}}$ & $1340$ & $3900$ & $8000$ \\ \hline	
	  $\mu_p \bracks{\frac{cm^2}{Vs}}$ & $460$ & $1900$ & $400$ \\ \hline	
	\end{longtable}
	\renewcommand{\arraystretch}{1}	
	
	\subsection{Konstanten} \label{ch:constants}
	\renewcommand{\arraystretch}{1.5}
	
	\begin{longtable} {|p{0.6cm}|p{4.4cm}|p{8.4cm}|} \hline
	% Definition des Tabellenkopfes auf der ersten Seite
	%Spaltenbezeichnungen
	\textbf{Ze.} & \textbf{Wert} & \textbf{Bedeutung}\\
	\hline
	\endfirsthead % Erster Kopf zu Ende
	% Definition des Tabellenkopfes auf den folgenden Seiten
	\caption{Konstanten}\\ \hline
	%Spaltenbezeichnungen
	\textbf{Ze.} & \textbf{Wert} & \textbf{bedeutung}\\
	\hline
	\endhead % Zweiter Kopf ist zu Ende
	\multicolumn{3}{r}{Fortsetzung auf Folgeseite}\\
	\endfoot
	\hline
	%\multicolumn{3}{r}{Ende} \\
	\endlastfoot
	
	%a-g
	$c$ & $2,998...\cdot 10^8 \bracks{frac{m}{s}}$ & Lichtgeschwindigkeit\\ \hline
	$e,q$ & $1,602176...\cdot 10^{-19}\bracks{C}$ & Elementarladung\\ \hline
	$e,q$ & $1,602176...\cdot 10^{-19}\bracks{J}$ & Elementarladung\\ \hline
	%h-n
	$h$ & $6,63 \cdot 10^{-34} \bracks{Js}$ & Planck-Konstante\\ \hline
	$h$ & $4,136...\cdot 10^{-15} \bracks{eVs}$ & Planck-Konstante\\ \hline
	$\hbar$ & $\frac{h}{2\pi}$ & Plancksches Wirkungsquantum\\ \hline
	$k$ & $8,6173 \cdot 10^{-5} \bracks{\frac{eV}{K}}$ & Boltzmann Konstante\\ \hline
	$kT$ & $25,85 \bracks{meV}$ & mit der Boltzmann Konstante und $T=300K$ \\ \hline
	%m-u
	$m_0$ & $9,11 \cdot 10^{-31} \bracks{kg}$ & Elektronenmasse\\ \hline
	 
	%v-z
	
	%griechisch
	$\varepsilon_0$ & $8,854..\cdot 10^{-12}\bracks{\frac{As}{Vm}}$ & Dielektrizitätskonstante des Vakuuums \\ \hline
	$\varepsilon_{Si}$ & $11,90$ & Korrekturfaktor Dielektrizitätskonstante für Silizium\\ \hline
	%Sonderzeichen
	\end{longtable}
	\renewcommand{\arraystretch}{1}

\nocite{*}
\bibliography{../bib/lit}

\end{document}
