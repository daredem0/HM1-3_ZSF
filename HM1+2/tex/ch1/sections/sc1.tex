\section{Trigonometrie}
	  \subsection{Winkelfunktionen}
	  \begin{align}
	    sin(\alpha) = \frac{Gegenkathete}{Hypothenuse}\\
	    cos(\alpha) = \frac{Ankathete}{Hypothenuse}\\
	    tan(\alpha) = \frac{Gegenkathete}{Ankathete} \label{eq:trigo_Winkelf}
	  \end{align}
	  
	  \subsubsection{Wichtige Werte}
	  \renewcommand{\arraystretch}{1.5}
	  \begin{tabular}{|p{3.2cm}|p{1.8cm}|p{1.8cm}|p{1.8cm}|p{1.8cm}|p{1.8cm}|}\hline
	  $\alpha$ in Gradmaß & $0^{\circ}$ & $30^{\circ}$ & $45^{\circ}$ & $60^{\circ}$ & $90^{\circ}$ \\ \hline
	  $\alpha$ in Bogenmaß & $0$ & $\frac{\Pi}{6}$ & $\frac{\pi}{4}$ & $\frac{\pi}{3}$ & $\frac{\pi}{2}$ \\ \hline
	  $sin\alpha$ & $\frac{1}{2}\sqrt{0}$ & $\frac{1}{2}\sqrt{1}$ & $\frac{1}{2} \sqrt{2}$ & $\frac{1}{2}\sqrt{3}$ & $1$ \\ \hline
	  $cos\alpha$ & $1$ & $\frac{1}{2}\sqrt{3}$ & $\frac{1}{2}\sqrt{2}$ & $\frac{1}{2}\sqrt{1}$ & $\frac{1}{2}\sqrt{0}$ \\ \hline
	  $tan\alpha$ & $0$ & $\frac{1}{3}\sqrt{3}$ & $1$ & $\sqrt{3}$ & n.d. \\ \hline
	  \end{tabular}
	  \renewcommand{\arraystretch}{1}
	  
	  \subsection{Sinussatz}
	  \begin{equation}
	    \frac{a}{sin\alpha} = \frac{b}{sin\beta} = \frac{c}{sin\gamma} = 2r = \frac{abc}{2F} \label{eq:allg_sinussatz}  
	  \end{equation}
	  
	  \subsection{Cosinussatz}
	  \begin{align}   
	    a^2 = b^2 + c^2 - 2bc cos\alpha\\
	    b^2 = c^2 + a^2 - 2ca cos\beta\\
	    c^2 = a^2 + b^2 - 2ab cos\gamma \label{eq:trigo_cosinussatz}
	  \end{align}
	  
	  \subsection{Tangenssatz}
	  \begin{align}
	    \frac{b + c}{b - c} = \frac{tan\left(\frac{\beta + \gamma}{2}\right)}{tan\left(\frac{\beta - \gamma}{2}\right)} 
	    = \frac{cot\left(\frac{\alpha}{2} \right)}{tan\left(\frac{\beta - \gamma}{2}\right)}
	  \end{align}
	  Analog für $\frac{a + b}{a - b}$ und $\frac{a + c}{a - c}$.\label{eq:trigo_tangenssatz}
	  
	  \subsection{Umwandlung}
	  \begin{align}
	    tan\alpha = \frac{sin\alpha}{cos\alpha}\\
	    sin^2(\alpha) + cos^2(\alpha) = 1\\
	    1 + tan^2(\alpha) = \frac{1}{cos^2(\alpha)} = sec^2(\alpha)\\
	    1 + cot^2(\alpha) = \frac{1}{sin^2(\alpha)} = csc^2(\alpha)\label{eq:trigo_umwandlung}
	  \end{align}
	  
	  \subsection{Additionstheoreme}
	  \begin{align}
	    sin(x \pm y) &= sin(x) cos(y) \pm cos(x) sin(y)\\
	    cos(x \pm y) &= cos(x) cos(y) \mp sin(x) sin(y)\\\\
	    tan(x \pm y) &= \frac{tan(x) \pm tan(y)}{1 \mp tan(x) tan(y)} = \frac{sin(x \pm y)}{cos(x \pm y)}\\
	    cot(x \pm y) &= \frac{cot(x) cot(y) \mp 1}{cot(y) \pm cot(x)} = \frac{cos(x \pm y)}{sin(x \pm y)}\\\\
	    sin(x + y) \cdot sin(x - y) &= cos^2(y) - cos^2(x) = sin^2(x) - sin^2(y)\\
	    cos(x + y) \cdot cos(x - y) &= cos^2(y) - sin^2(x) = cos^2(x) - sin^2(y)\label{eq:trigo_addtheo}
	  \end{align}
	  
	  \subsection{Folgerungen aus den Additionstheoremen}
	  \begin{align}
	  cos^2(\frac{x}{2}) + sin^2(\frac{x}{2}) &\;\;= cos(\frac{x}{2}) cos(\frac{x}{2}) + sin(\frac{x}{2}) sin(\frac{x}{2})\\ 
	  &\overset{\eqref{eq:trigo_addtheo}}{=} cos(\frac{x}{2}-\frac{x}{2}) = cos(0) = 1\\
	  \\
	  2sin(\frac{x}{2})cos(\frac{x}{2}) &\;\;= sin(\frac{x}{2})cos(\frac{x}{2}) + sin(\frac{x}{2})cos(\frac{x}{2})\\
	  &\overset{\eqref{eq:trigo_addtheo}}{=} sin(\frac{x}{2}+ \frac{x}{2}) = sin(x)\\
	  \\
	  sin(2x) = sin(x+x) &\overset{\eqref{eq:trigo_addtheo}}{=} sin(x)cos(x) + sin(x)cos(x) \\
	  &\;\;=2sin(x)cos(x) \label{eq:trigo:addtheo_folg}
	  \end{align}
		
	