\section{Extremalprobleme}
Es sei $f:\R \rightarrow \R$, dann ist $f'(x^*) = 0$ notenwide Bedingung dafür, dass ein Minimum oder Maximum vorliegt, sofern $f$ diffbar ist.
\begin{satz}
  \textbf{Globale Theorie: } $[a,b]$ sei ein abgeschlossenes Intervall. $f:[a,b] \rightarrow \R$ sei stetig. Dann gibt es je ein $\underline{x}, \overline{x} \in [a,b]$ mit $f(\underline{x}) = \min\limits_{x \in [a,b]}f(x)$ und $f(\overline{x}) = \max\limits_{x \in [a,b]}f(x)$.\label{ax:extrema_global}
\end{satz}
\begin{bem} (Zu \eqref{ax:extrema_global})
  Stetige Funktionen auf kompakten Mengen nehmen das Maximum und Minimum an.
\end{bem}
\begin{satz}
  \textbf{Lokale Theorie: } Sei $f:[a,b] \rightarrow \R$ mit $f\in \C^3$ (also 3 mal stetig diffbar). Für $x* \in (a,b)$ gilt dann:
  \begin{itemize}
    \item[a) ] Aus $f'(x^*) = 0$ und $f''(x^*) > 0$ folgt, dass $f$ in $x^*$ ein lokales Minimum hat.
    \item[b) ] Aus $f'(x^*) = 0$ und $f''(x^*) < 0$ folgt, dass $f$ in $x^*$ ein lokales Maximum hat.
  \end{itemize}
\end{satz}
\newpage