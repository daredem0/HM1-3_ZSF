\section{Lineare skalare Differentialgleichungen mit konstanten Koeffizienzen}
  \begin{definition}
    Lineare skalare Diff.gleichungen mit konst. Koeffizienten sind Gleichungen der bauart:
    \begin{align*}
      Ly(x) = a_n y^{(n)}(x)+a_{n-1}y^{(n-1)}(x)+...+a_1y'(x) + a_o y(x) = f(x)
    \end{align*}
  \end{definition}
  \subsection{Ansätze}
  \begin{itemize}
    \item[a)] Homogene Diff.gleichung:
    Mit $\lambda$ als einfache NST:
    \begin{equation}
      y(x) = e^{\displaystyle\lambda x} \label{eq:dgl_Ansatz_a}
    \end{equation}
    Mit $\lambda$ als $n$-fache NST:
    \begin{equation}
      f(x) = e^{\displaystyle\lambda x}, xe^{\displaystyle\lambda x}, ..., x^{n-1}e^{\displaystyle\lambda x} 
    \end{equation}
    Mit $\lambda = \alpha + \beta i$ (komplexe Nullstellen):
    \begin{equation}
      e^{\alpha x}cos(\beta x),\quad e^{\alpha x} sin(\beta x)
    \end{equation}
    \item[b)] Inhomogenität der Form $f(x) = p_k(x) e^{sx}$ mit $p_k(x)$ als ein Polynom $k$-ten Grades.
    \begin{equation}
      y_p(x) = R_k(x)e^{sx}x^q
    \end{equation}
    Mit $R_k(x) = a_kx^k+...+a_0$ und $q$ als Vielfachheit der Nullstelle $s$ (ist $s$ keine Nullstelle so ist $q = 0$ und damit $x^q = 1$).
    \item[c)] Inhomogenität der Form $cos(kx)$ oder $sin(kx)$.
    \begin{equation}
      yp(x) = (a cos(kx) + b sin(kx)) x^q
    \end{equation}
    Beispiel Umformung:
    \begin{align}
      f(x) &= sin(4x)\nonumber\\
      Mit \quad
      (1)\;\;\; e^{i\varphi} &= cos \varphi + i sin\varphi\nonumber \\
      und \quad
      (2)\; e^{-i\varphi} &= cos \varphi - i sin \varphi\nonumber\\
      folgt \;mit \;(1) + (2) \;&bzw.\; (1)-(2)\nonumber\\
      e^{i\varphi} + e^{-i\varphi} = 2\;cos\varphi \quad &bzw. \quad e^{i\varphi}-e^{-i\varphi} = 2 \; sin\varphi\nonumber\\
      \Rightarrow cos \varphi = \frac{1}{2} (e^{i\varphi}+e^{-i\varphi}) &\quad \Rightarrow sin \varphi = \frac{1}{2i} (e^{i\varphi}-e^{-i\varphi}),\quad \forall \varphi \in \R
    \end{align}
    $q$ im Ansatz gibt die Vielfachheit der NST $i \varphi$ im charackteristischen Polynom des homogenen Teils der Gleichung an.
    \item[d)] Inhomogenität der Form $q_k(x)e^{\alpha x} cos(\beta x)$ oder $q_k(x)e^{\alpha x} sin(\beta x)$
    \begin{equation}
      yp(x) = R_k(x)x^qe^{\alpha x}cos(\beta x) + \tilde{R}_k(x) x^qe^{\alpha x}cos{\beta x}
    \end{equation}
  \end{itemize}
  \subsection{Vorgehensweise}
    \begin{flalign*}
      &\textbf{Schritt 1: } \text{Ansatz wählen}&
    \end{flalign*}
    \begin{flalign*}
      &\textbf{Schritt 2: } \text{Ansatz einsetzen und charackteristisches Polynom bilden}& \\
      &\text{Beispiel:}&
    \end{flalign*}
    \begin{align*}
      &y'' + 3y'+2y = 0 \\
      &\overset{\eqref{eq:dgl_Ansatz_a}}{\Rightarrow} \left(e^{\lambda x}\right)'' + 3\left(e^{\lambda x}\right)' + 2e^{\lambda x} = 0 \\
      \left(e^{\lambda x}\right)' &\;\;= \lambda e^{\lambda x} \Rightarrow \left(e^{\lambda x}\right)'' = \lambda^2 e^{\lambda x} \\
      &\;\;\Rightarrow \lambda^2 + 3 \lambda + 2 = 0
    \end{align*}
    \begin{flalign*}
      &\textbf{Schritt 3.1: } \text{Nullstellen des char. Polynoms suchen und in Ansatz einsetzen}&\\
      &Beispiel:&
    \end{flalign*}
    \begin{align*}
      \lambda^2 + 3 \lambda + 2 = 0 \Rightarrow \lambda_1 = -2,\quad \lambda_2 = -1
    \end{align*}
    \begin{flalign*}
      &\textbf{Schritt 3.2: } \text{Gegebenenfalls komplexe NST in reale umwandeln}&\\
      &Beispiel:&
    \end{flalign*}
    \begin{align*}
      y(x) &= C_1 e^{-x+ix} + C_2 e^{-x-ix} \\
      \Rightarrow y(x) &= \tilde{C}_1e^{-x}cosx + \tilde{C}_2 e^{-x}sinx \qquad, \tilde{C}_1, \tilde{C}_2 \in \R
    \end{align*}
    \begin{flalign*}
      &\textbf{Schritt 4: } \text{Allgemeine Lösung aufstellen}&
    \end{flalign*}
    \begin{align*}
      \Rightarrow y(x) = C_1 e^{-2x} + C_2 e^{-x}\quad, C_1,C_2\in \R
    \end{align*}
    
    \newpage
    \subsubsection{Anfangswertproblem}
    Falls Anfangswerte vorhanden sind können an dieser Stelle die Konstanten Koeffizienten explizit bestimmt werden. \newline
    \begin{flalign*}
    &Beispiel:&
    \end{flalign*}
    \begin{align*}
	    y(0) &= 1,\;y'(0) = 0\\
	    y(t) &= C_1 e^{5t} + C_2 e^{-2t}\\
	    &\Rightarrow y(0) = C_1 e^{5\cdot 0} + C_2 e^{-2\cdot 0} = 1\\ 
      &\Rightarrow C_1 = 1-C_2\\
      y'(t)& = 5 C_1 e^{5t} -2 C_2 e^{-2t} \Rightarrow y'(0) = 5C_1 - 2C_2 = 0\\
      &\Rightarrow y'(0) = 0 = 5(1-C_2)-2C_2 \Rightarrow C_2 = \frac{5}{7}\\
      &\Rightarrow C_1 = 1-C_2 = 1- \frac{5}{7} = \frac{2}{7}\\
      \; \\
      &\Rightarrow y(t) = \frac{2}{7} e^{5t}+\frac{5}{7}e^{-2t}
    \end{align*}
    
    \subsubsection{Inhomogenität}
      \begin{flalign*}
        &\textbf{Schritt 1: } \text{Ansatz wählen}&
      \end{flalign*}
      \begin{flalign*}
        &\textbf{Schritt 2: } \text{Ansatz gegebenenfalls ableiten und in homogenen Teil einsetzen}&
      \end{flalign*}    
      \begin{flalign*}
        &\textbf{Schritt 3: } \text{Über Koeffizientenvergleich Vorfaktoren bestimmen}&
      \end{flalign*}    
      \begin{flalign*}
        &\textbf{Schritt 4: } \text{Allgemeine Lösung bilden}&
      \end{flalign*}    
      \begin{equation}
        y(x) = y_{hom}(x) + y_p(x)
      \end{equation}
      \begin{flalign*}
        &\textbf{Schritt 5: } \text{Gegebenenfalls Anfangswertproblem lösen}&
      \end{flalign*}    
      \newpage
    \section{Lineare skalare Differentialgleichungen mit konstanten Koeffizienzen}
  \begin{definition}
    Lineare skalare Diff.gleichungen mit konst. Koeffizienten sind Gleichungen der bauart:
    \begin{align*}
      Ly(x) = a_n y^{(n)}(x)+a_{n-1}y^{(n-1)}(x)+...+a_1y'(x) + a_o y(x) = f(x)
    \end{align*}
  \end{definition}
  \subsection{Ansätze}
  \begin{itemize}
    \item[a)] Homogene Diff.gleichung:
    Mit $\lambda$ als einfache NST:
    \begin{equation}
      y(x) = e^{\displaystyle\lambda x} \label{eq:dgl_Ansatz_a}
    \end{equation}
    Mit $\lambda$ als $n$-fache NST:
    \begin{equation}
      f(x) = e^{\displaystyle\lambda x}, xe^{\displaystyle\lambda x}, ..., x^{n-1}e^{\displaystyle\lambda x} 
    \end{equation}
    Mit $\lambda = \alpha + \beta i$ (komplexe Nullstellen):
    \begin{equation}
      e^{\alpha x}cos(\beta x),\quad e^{\alpha x} sin(\beta x)
    \end{equation}
    \item[b)] Inhomogenität der Form $f(x) = p_k(x) e^{sx}$ mit $p_k(x)$ als ein Polynom $k$-ten Grades.
    \begin{equation}
      y_p(x) = R_k(x)e^{sx}x^q
    \end{equation}
    Mit $R_k(x) = a_kx^k+...+a_0$ und $q$ als Vielfachheit der Nullstelle $s$ (ist $s$ keine Nullstelle so ist $q = 0$ und damit $x^q = 1$).
    \item[c)] Inhomogenität der Form $cos(kx)$ oder $sin(kx)$.
    \begin{equation}
      yp(x) = (a cos(kx) + b sin(kx)) x^q
    \end{equation}
    Beispiel Umformung:
    \begin{align}
      f(x) &= sin(4x)\nonumber\\
      Mit \quad
      (1)\;\;\; e^{i\varphi} &= cos \varphi + i sin\varphi\nonumber \\
      und \quad
      (2)\; e^{-i\varphi} &= cos \varphi - i sin \varphi\nonumber\\
      folgt \;mit \;(1) + (2) \;&bzw.\; (1)-(2)\nonumber\\
      e^{i\varphi} + e^{-i\varphi} = 2\;cos\varphi \quad &bzw. \quad e^{i\varphi}-e^{-i\varphi} = 2 \; sin\varphi\nonumber\\
      \Rightarrow cos \varphi = \frac{1}{2} (e^{i\varphi}+e^{-i\varphi}) &\quad \Rightarrow sin \varphi = \frac{1}{2i} (e^{i\varphi}-e^{-i\varphi}),\quad \forall \varphi \in \R
    \end{align}
    $q$ im Ansatz gibt die Vielfachheit der NST $i \varphi$ im charackteristischen Polynom des homogenen Teils der Gleichung an.
    \item[d)] Inhomogenität der Form $q_k(x)e^{\alpha x} cos(\beta x)$ oder $q_k(x)e^{\alpha x} sin(\beta x)$
    \begin{equation}
      yp(x) = R_k(x)x^qe^{\alpha x}cos(\beta x) + \tilde{R}_k(x) x^qe^{\alpha x}cos{\beta x}
    \end{equation}
  \end{itemize}
  \subsection{Vorgehensweise}
    \begin{flalign*}
      &\textbf{Schritt 1: } \text{Ansatz wählen}&
    \end{flalign*}
    \begin{flalign*}
      &\textbf{Schritt 2: } \text{Ansatz einsetzen und charackteristisches Polynom bilden}& \\
      &\text{Beispiel:}&
    \end{flalign*}
    \begin{align*}
      &y'' + 3y'+2y = 0 \\
      &\overset{\eqref{eq:dgl_Ansatz_a}}{\Rightarrow} \left(e^{\lambda x}\right)'' + 3\left(e^{\lambda x}\right)' + 2e^{\lambda x} = 0 \\
      \left(e^{\lambda x}\right)' &\;\;= \lambda e^{\lambda x} \Rightarrow \left(e^{\lambda x}\right)'' = \lambda^2 e^{\lambda x} \\
      &\;\;\Rightarrow \lambda^2 + 3 \lambda + 2 = 0
    \end{align*}
    \begin{flalign*}
      &\textbf{Schritt 3.1: } \text{Nullstellen des char. Polynoms suchen und in Ansatz einsetzen}&\\
      &Beispiel:&
    \end{flalign*}
    \begin{align*}
      \lambda^2 + 3 \lambda + 2 = 0 \Rightarrow \lambda_1 = -2,\quad \lambda_2 = -1
    \end{align*}
    \begin{flalign*}
      &\textbf{Schritt 3.2: } \text{Gegebenenfalls komplexe NST in reale umwandeln}&\\
      &Beispiel:&
    \end{flalign*}
    \begin{align*}
      y(x) &= C_1 e^{-x+ix} + C_2 e^{-x-ix} \\
      \Rightarrow y(x) &= \tilde{C}_1e^{-x}cosx + \tilde{C}_2 e^{-x}sinx \qquad, \tilde{C}_1, \tilde{C}_2 \in \R
    \end{align*}
    \begin{flalign*}
      &\textbf{Schritt 4: } \text{Allgemeine Lösung aufstellen}&
    \end{flalign*}
    \begin{align*}
      \Rightarrow y(x) = C_1 e^{-2x} + C_2 e^{-x}\quad, C_1,C_2\in \R
    \end{align*}
    
    \newpage
    \subsubsection{Anfangswertproblem}
    Falls Anfangswerte vorhanden sind können an dieser Stelle die Konstanten Koeffizienten explizit bestimmt werden. \newline
    \begin{flalign*}
    &Beispiel:&
    \end{flalign*}
    \begin{align*}
	    y(0) &= 1,\;y'(0) = 0\\
	    y(t) &= C_1 e^{5t} + C_2 e^{-2t}\\
	    &\Rightarrow y(0) = C_1 e^{5\cdot 0} + C_2 e^{-2\cdot 0} = 1\\ 
      &\Rightarrow C_1 = 1-C_2\\
      y'(t)& = 5 C_1 e^{5t} -2 C_2 e^{-2t} \Rightarrow y'(0) = 5C_1 - 2C_2 = 0\\
      &\Rightarrow y'(0) = 0 = 5(1-C_2)-2C_2 \Rightarrow C_2 = \frac{5}{7}\\
      &\Rightarrow C_1 = 1-C_2 = 1- \frac{5}{7} = \frac{2}{7}\\
      \; \\
      &\Rightarrow y(t) = \frac{2}{7} e^{5t}+\frac{5}{7}e^{-2t}
    \end{align*}
    
    \subsubsection{Inhomogenität}
      \begin{flalign*}
        &\textbf{Schritt 1: } \text{Ansatz wählen}&
      \end{flalign*}
      \begin{flalign*}
        &\textbf{Schritt 2: } \text{Ansatz gegebenenfalls ableiten und in homogenen Teil einsetzen}&
      \end{flalign*}    
      \begin{flalign*}
        &\textbf{Schritt 3: } \text{Über Koeffizientenvergleich Vorfaktoren bestimmen}&
      \end{flalign*}    
      \begin{flalign*}
        &\textbf{Schritt 4: } \text{Allgemeine Lösung bilden}&
      \end{flalign*}    
      \begin{equation}
        y(x) = y_{hom}(x) + y_p(x)
      \end{equation}
      \begin{flalign*}
        &\textbf{Schritt 5: } \text{Gegebenenfalls Anfangswertproblem lösen}&
      \end{flalign*}    
      \newpage
    