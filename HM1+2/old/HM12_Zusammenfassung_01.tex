%%%%%Präambel%%%%%

\documentclass[12pt,a4paper]{article}%Schriftgröße, Papierformat einstellen
%\documentclass{scrbook}
\usepackage[top=30mm,bottom=30mm]{geometry}
\usepackage{lipsum}
%Pakete laden zur deutschen Rechtschreibung und für Umlaute
\usepackage[T1]{fontenc}
\usepackage[ngerman]{babel}
\usepackage[utf8]{inputenc} %für Windows, Linux
%\usepackage[applemac]{inputenc} %für Mac
%\usepackage{xcolor}
\usepackage[dvipsnames]{xcolor}
\usepackage{cancel}
\usepackage{titlesec}
\usepackage{cite}
\usepackage{filecontents}
\usepackage{tabularx}
\usepackage{harvard}
\usepackage{units}
\usepackage{longtable} 
\usepackage{chngcntr}
\usepackage{stmaryrd}
\usepackage{array}
\let\harvardleftorig\harvardleft
%\usepackage[round]{natbib}
%\usepackage{hyperref}
\usepackage[nottoc,numbib]{tocbibind}

%Pakete laden zu mathematischen Symbolen etc.
\usepackage{calc} 
\usepackage{amsmath,amssymb,amsthm}
\usepackage{scrpage2}
\pagestyle{scrheadings}
\clearscrheadfoot
\automark[chapter]{section}
\ofoot{\pagemark}
\ifoot{Florian Leuze}
\chead{\headmark}
\setfootsepline{1pt}
\setheadsepline{1pt}
%\setheadsepline[\textwidth+20pt]{0.5pt}

%Inhaltsverzeichnis mit Links erstellen
\usepackage[colorlinks,
pdfpagelabels,
pdfstartview = FitH,
bookmarksopen = true,
bookmarksnumbered = true,
linkcolor = black,
plainpages = false,
hypertexnames = false,
citecolor = black] {hyperref}

% Umgebungen für Definitionen, Sätze, usw.
\newtheorem{satz}{Satz}[section]
\newtheorem{definition}[satz]{Definition}     
\newtheorem{lemma}[satz]{Lemma}	
% Es werden Sätze, Definitionen etc innerhalb einer Section mit
% 1.1, 1.2 etc durchnummeriert, ebenso die Gleichungen mit (1.1), (1.2) ..                  
\numberwithin{equation}{section}

\setcounter{secnumdepth}{4}
\setcounter{tocdepth}{4}

\titleformat{\paragraph}
{\normalfont\normalsize\bfseries}{\theparagraph}{1em}{}
\titlespacing*{\paragraph}
{0pt}{3.25ex plus 1ex minus .2ex}{1.5ex plus .2ex}

%neue Befehle definieren
\newcommand{\R}{\mathbb{R}} %zB \R als Abkürzung für das Symbol der reellen Zahlen
\newcommand{\N}{\mathbb{N}}
\newcommand{\Z}{\mathbb{Z}}
\newcommand{\Q}{\mathbb{Q}}
\newcommand{\C}{\mathbb{C}}

\newcommand{\subsubsubsection}{\paragraph}
\newcommand\citevgl
{\def\harvardleft{(vgl.\ \global\let\harvardleft\harvardleftorig}%
 \cite
}
\newcommand\citeVgl
{\def\harvardleft{(Vgl.\ \global\let\harvardleft\harvardleftorig}%
 \cite
}

\def\ccite#1#2{\glqq #1\grqq\cite{#2}}

\newcolumntype{L}[1]{>{\raggedleft\let\newline\\\arraybackslash\hspace{0pt}}m{#1}}
%Makros
%Makro Color
%#1 Text
\def\colBord#1{\begingroup\color{Fuchsia}{#1}\endgroup}
\def\colRed#1{\begingroup\color{Red}{#1}\endgroup}
\def\colGreen#1{\begingroup\color{LimeGreen}{#1}\endgroup}
\def\colBlue#1{\begingroup\color{NavyBlue}{#1}\endgroup}

\def\usGreen#1#2{\underset{\colGreen{#1}}{#2}}
\def\usBord#1#2{\underset{\colBord{#1}}{#2}}

\def\ubGreen#1#2{\underbrace{#2}_{\colGreen{#1}}}

\def\defF{\textbf{Def.: }}
\def\mDef#1{
\begin{definition}
  #1
\end{definition}}

\def\vecT#1{\left(\begin{array}{c} #1 \end{array}\right)}
\def\dddot{\cdot \\ \cdot \\ \cdot}
\def\vecD#1{\vecT{#1_1 \\ \dddot \\ #1_d}}
\def\vecDt#1#2{\vecT{#1 \\ \dddot \\ #2}}
\def\vecN{\mathcal{O}}
\def\vspan#1{span \lbrace #1 \rbrace}
\def\vdim#1{dim \lbrace #1 \rbrace}
\def\vker#1{ker \lbrace #1 \rbrace}
\def\vrang#1{Rang \lbrace #1 \rbrace}

\def\epsF{\pmb{\varepsilon}}

\def\multiTwo#1#2{\multicolumn{2}{>{\hsize=\dimexpr2\hsize+2\tabcolsep+\arrayrulewidth\relax}#1}{#2}}
\def\multiThree#1#2{\multicolumn{3}{>{\hsize=\dimexpr3\hsize+4\tabcolsep+2\arrayrulewidth\relax}#1}{#2}}

\def\inR#1{\qquad ,\; #1 \in \R}
\def\inRs{\in \R}
\def\bracks#1{\left[ #1 \right]}
\def\abs#1{\left| #1 \right|}
\def\brac#1{\left( #1 \right)}

%laziness
\def\fermi{Fermi-Dirac-Verteilung}

\newcolumntype{L}[1]{>{\raggedleft\let\newline\\\arraybackslash\hspace{0pt}}m{#1}}
\newcolumntype{R}[1]{>{\raggedright\let\newline\\\arraybackslash\hspace{0pt}}m{#1}}



\def\formTab#1#2{
\begin{equation}
  \begin{tabularx}{12cm}{R{3cm} l l}
    #1 &: &$#2$
  \end{tabularx}
\end{equation}
}
\newcommand{\formTabL}[3]{
\begin{equation}
  \begin{tabularx}{12cm}{R{3cm} l l}
    #1 &: &$#2$ 
  \end{tabularx}
  \label{eq:#3}
\end{equation}}
\def\formTn{$ \\ $\;$ & $\;$ & $}
\def\formTnQ{$ \\ $\;$ & $\;$ & $\qquad}
\def\formTnQQ{$ \\ $\;$ & $\;$ & $\qquad \qquad}
\def\formTnQQQ{$ \\ $\;$ & $\;$ & $\qquad \qquad \qquad}

\renewcommand{\theequation}{\arabic{section}.\arabic{subsection}
.\arabic{equation}}
%Setzt den equation-Zaehler nach jeder Seite zurueck
\numberwithin{equation}{subsection}	


%\setlength\abovedisplayskip{0pt}

% Auf der Seite http://detexify.kirelabs.org/classify.html können Sie mathematische Symbole, Pfeile usw per Maus eingeben und bekommen den Latex-Befehl dafür angezeigt.
% detexify gibt es auch als App...

%jetzt beginnt das eigentliche Dokument
\begin{document}

\bibliographystyle{agsm}

\author{}
\title{\underline{HM1-2 Zusammenfassung} \\ $\;$ \\ $\;$ \\ Florian Leuze}
\date{}

\maketitle % erzeugt den Kopf
\newpage
\section{\underline{Inhalt}}
\tableofcontents

  \subsection{Versionierung}
  \begin{tabular}{|p{2cm}|p{1cm}|p{1.5cm}|p{8.5cm}|}\hline
    Datum & Vers. & Kürzel & Änderung \\ \hline
    19.04.2018 & 0.1 & FL & Erzeugung Dokument; Erzeugung Inhaltsverzeichnis; Erzeugung Versionierung; Erzeugung 2.1 - 2.7.4 \\ \hline
    19.04.2018 & 0.2 & FL & Korrekturen 2.6.1 - 2.6.9 u. 2.7.1 - 2.7.2 Titel\\ \hline
    20.04.2018 & 0.2.1 & FL & Erzeugung 2.7.1.1 - 2.7.1.4; Korrektur Riemannsche Untersumme; Erzeugung Literaturverzeichnis \\ \hline
    01.05.2018 & 0.2.2 & FL & Neustrukturierung; Erzeugung Allgemeines; Erzeugung Zahlen \\ \hline
    01.08.2018 & 0.3.0 & FL & Überarbeitung Trigonometrie \\ \hline
  \end{tabular}

\newpage

\section{Allgemeines}
	\subsection{Trigonometrie}
	  \subsubsection{Winkelfunktionen}
	  \begin{align}
	    sin(\alpha) = \frac{Gegenkathete}{Hypothenuse}\\
	    cos(\alpha) = \frac{Ankathete}{Hypothenuse}\\
	    tan(\alpha) = \frac{Gegenkathete}{Ankathete} \label{eq:trigo_Winkelf}
	  \end{align}
	  
	  \subsubsubsection{Wichtige Werte}
	  \renewcommand{\arraystretch}{1.5}
	  \begin{tabular}{|p{3.2cm}|p{1.8cm}|p{1.8cm}|p{1.8cm}|p{1.8cm}|p{1.8cm}|}\hline
	  $\alpha$ in Gradmaß & $0^{\circ}$ & $30^{\circ}$ & $45^{\circ}$ & $60^{\circ}$ & $90^{\circ}$ \\ \hline
	  $\alpha$ in Bogenmaß & $0$ & $\frac{\Pi}{6}$ & $\frac{\pi}{4}$ & $\frac{\pi}{3}$ & $\frac{\pi}{2}$ \\ \hline
	  $sin\alpha$ & $\frac{1}{2}\sqrt{0}$ & $\frac{1}{2}\sqrt{1}$ & $\frac{1}{2} \sqrt{2}$ & $\frac{1}{2}\sqrt{3}$ & $1$ \\ \hline
	  $cos\alpha$ & $1$ & $\frac{1}{2}\sqrt{3}$ & $\frac{1}{2}\sqrt{2}$ & $\frac{1}{2}\sqrt{1}$ & $\frac{1}{2}\sqrt{0}$ \\ \hline
	  $tan\alpha$ & $0$ & $\frac{1}{3}\sqrt{3}$ & $1$ & $\sqrt{3}$ & n.d. \\ \hline
	  \end{tabular}
	  \renewcommand{\arraystretch}{1}
	  
	  \subsubsection{Sinussatz}
	  \begin{equation}
	    \frac{a}{sin\alpha} = \frac{b}{sin\beta} = \frac{c}{sin\gamma} = 2r = \frac{abc}{2F} \label{eq:allg_sinussatz}  
	  \end{equation}
	  
	  \subsubsection{Cosinussatz}
	  \begin{align}   
	    a^2 = b^2 + c^2 - 2bc cos\alpha\\
	    b^2 = c^2 + a^2 - 2ca cos\beta\\
	    c^2 = a^2 + b^2 - 2ab cos\gamma \label{eq:trigo_cosinussatz}
	  \end{align}
	  
	  \subsubsection{Tangenssatz}
	  \begin{align}
	    \frac{b + c}{b - c} = \frac{tan\left(\frac{\beta + \gamma}{2}\right)}{tan\left(\frac{\beta - \gamma}{2}\right)} 
	    = \frac{cot\left(\frac{\alpha}{2} \right)}{tan\left(\frac{\beta - \gamma}{2}\right)}
	  \end{align}
	  Analog für $\frac{a + b}{a - b}$ und $\frac{a + c}{a - c}$.\label{eq:trigo_tangenssatz}
	  
	  \subsubsection{Umwandlung}
	  \begin{align}
	    tan\alpha = \frac{sin\alpha}{cos\alpha}\\
	    sin^2(\alpha) + cos^2(\alpha) = 1\\
	    1 + tan^2(\alpha) = \frac{1}{cos^2(\alpha)} = sec^2(\alpha)\\
	    1 + cot^2(\alpha) = \frac{1}{sin^2(\alpha)} = csc^2(\alpha)\label{eq:trigo_umwandlung}
	  \end{align}
	  
	  \subsubsection{Additionstheoreme}
	  \begin{align}
	    sin(x \pm y) &= sin(x) cos(y) \pm cos(x) sin(y)\\
	    cos(x \pm y) &= cos(x) cos(y) \mp sin(x) sin(y)\\\\
	    tan(x \pm y) &= \frac{tan(x) \pm tan(y)}{1 \mp tan(x) tan(y)} = \frac{sin(x \pm y)}{cos(x \pm y)}\\
	    cot(x \pm y) &= \frac{cot(x) cot(y) \mp 1}{cot(y) \pm cot(x)} = \frac{cos(x \pm y)}{sin(x \pm y)}\\\\
	    sin(x + y) \cdot sin(x - y) &= cos^2(y) - cos^2(x) = sin^2(x) - sin^2(y)\\
	    cos(x + y) \cdot cos(x - y) &= cos^2(y) - sin^2(x) = cos^2(x) - sin^2(y)\label{eq:trigo_addtheo}
	  \end{align}
	  
	  \subsubsection{Folgerungen aus den Additionstheoremen}
	  \begin{align}
	  cos^2(\frac{x}{2}) + sin^2(\frac{x}{2}) &\;\;= cos(\frac{x}{2}) cos(\frac{x}{2}) + sin(\frac{x}{2}) sin(\frac{x}{2})\\ 
	  &\overset{\eqref{eq:trigo_addtheo}}{=} cos(\frac{x}{2}-\frac{x}{2}) = cos(0) = 1\\
	  \\
	  2sin(\frac{x}{2})cos(\frac{x}{2}) &\;\;= sin(\frac{x}{2})cos(\frac{x}{2}) + sin(\frac{x}{2})cos(\frac{x}{2})\\
	  &\overset{\eqref{eq:trigo_addtheo}}{=} sin(\frac{x}{2}+ \frac{x}{2}) = sin(x)\\
	  \\
	  sin(2x) = sin(x+x) &\overset{\eqref{eq:trigo_addtheo}}{=} sin(x)cos(x) + sin(x)cos(x) \\
	  &\;\;=2sin(x)cos(x) \label{eq:trigo:addtheo_folg}
	  \end{align}

\section{Zahlen}
  \subsection{Zahlbereiche}
    \subsubsection{Natürliche Zahlen}
    \begin{itemize}
      \item Direkt vom Zählen abgeleitet.
      \item $\N = \{1, 2, 3, 4, ...\}$
      \item Gleichungen wie $n+x = m$ i.A. in $\N$ nicht lösbar. 
    \end{itemize}

    \subsubsection{Ganze Zahlen}
    \begin{itemize}
      \item $\Z = \{..., -3, -2, -1, 0, 1, 2, 3, ...\}$
      \item Gleichungen wie $n \cdot x = m$ i.A. in $\Z$ nicht lösbar.
    \end{itemize}

    \subsubsection{Rationale Zahlen}
		\begin{itemize} 
			\item $\Q = \left\lbrace \frac{m}{n}: n\in \N, m \in \Z\text{, m, n teilerfremd} \right\rbrace$
			\item Gleichungen wie $x^2 = 2$ in $\Q$ nicht lösbar.
		\end{itemize}

		\subsubsection{Reelle Zahlen}
		\begin{itemize}
			\item $\R = \left\lbrace x = \sum_{j = - \infty}^n x_j 10^j : x_j \in \{0, 1, ..., 9\}\; \text{für ein n} \in \Z \right\rbrace $
			\item Gleichungen wie $x^2 = -1$ in $\R$ nicht lösbar.
		\end{itemize}

		\subsubsection{Komplexe Zahlen}
		\begin{itemize}
			\item $\C = \left\lbrace z = x + iy:\; x,\;y \in \R \right\rbrace$
			\ $i$ heißt imaginäre Einheit, es gilt $\R \subset \C$.
		\end{itemize}
	
	\subsection{Algebraische Strukturen}
		\subsubsection{Gruppe}
		\defF \glqq Eine Menge $M$ mit der Verknüpfung $+:M\times M \rightarrow M$, deren Elemente die Eigenschaften $(G1) - (G3)$ erfüllen, heißt Gruppe. Gilt zusätzlich $(G4)$, so heißt $(M,+)$ eine kommutative Gruppe.\grqq \cite{HM12}
		In der Literatur findet man als alternative Bezeichnung kommutativ auch abelsch. \citeVgl{LinAF}
		
			\subsubsubsection{Gruppenaxiome}
		  \begin{equation}
			  \begin{tabularx}{14.7cm}{l l l}
					(\textbf{G1}) & $x+(y+z) = (x+y)+z$ & \colBlue{(Assoziativgesetz)}\\
					(\textbf{G2}) & $x+0 = x = 0+x$ & \colBlue{(0 ist das neutrale Element)}\\
					(\textbf{G3}) & $x+(-x) = (-x)+x = 0$ & \colBlue{(-x ist das inverse Element zu x)}\\
					(\textbf{G4}) & $x+y = y+x$ & \colBlue{(Kommutativität also Vertauschbarkeit)}\\
			  \end{tabularx}
			  \label{ax:groupaxioms}
			\end{equation}
			\newline
			\defF \glqq Sei eine Gruppe eine Verknüpfung mit $\cdot$ und $G' \subset G$ eine nichtleere Teilmenge. $G'$ heißt eine Untergruppe, wenn für $a,\;b \in G'$ auch $a\cdot b \in G'$ und $a^{-1} \in G'$. \cite{LinAF}
			\newline
			Im Bezug auf Gruppen und Ringe spielen die Begrifflichkeiten Isomorphismus und Homomorphismus eine Rolle. Beide Begriffe leiten sich von Morphismus (Struktur bzw. Form), Homo (gleich im Sinne von ähnlich) und Iso (gleich im Sinne von identisch) ab. 
			\newline \newline
			\defF \glqq Sind $G$ und $H$ Gruppen mit Verknüpfungen $\cdot$ und $\times$, so heißt eine Abbildung $\varphi : G \rightarrow H$ Homomorphismus (von Gruppen), wenn
			
			\begin{equation}
			  \varphi (a \cdot b) = \varphi (a) \times \varphi (b) \qquad  \forall a,\;b \in G\text{. \cite{LinAF}}
			\end{equation}
			\newline
			Ein Homomorphismus heißt Isomorphismus wenn er bijektiv ist. \cite{LinAF}
		
		\subsubsection{Ring}
		Erfüllt eine Menge die Eigenschaften einer Gruppe, hat jedoch zwei Verknüpfungen (z.B. $(+,-)$) spricht man von einem Ring.
		
		\subsubsection{Körper}
		Erfüllt eine Menge die Eigenschaften eines Ringes und ist zusätzlich kommutativ spricht man von einem Körper. Ein Beispiel für einen Körper ist die algebraische Struktur der rationalen und reellen Zahlen. 
		
			\subsubsubsection{Körperaxiome}
			
			\begin{equation}
				\begin{tabularx}{14.7cm}{l l l}
					(\textbf{K1}) & $x\cdot(y\cdot z) = (x\cdot y)\cdot z$ & \colBlue{(Assoziativgesetz)}\\
					(\textbf{K2}) & $x\cdot 1 = 1 \cdot x = x$ & \colBlue{(1 ist das neutrale Element)}\\
					(\textbf{K3}) & $x\cdot(\frac{1}{x}) = (\frac{1}{x}) \cdot x = 1$ & \colBlue{($\frac{1}{x}$ ist das inverse Element zu x)}\\
					(\textbf{K4}) & $x\cdot y = y\cdot x$ & \colBlue{(Kommutativität also Vertauschbarkeit)} \\
					(\textbf{D}) & $x \cdot (y+z) = x \cdot y + x \cdot z$ & \colBlue{(Kommutativität also Vertauschbarkeit)}\\
				\end{tabularx}
			\label{ax:körperaxiome}
			\end{equation}
			\newline
			\defF \glqq Eine Menge $M$ mit den Verknüpfungen $+:M\times M  \rightarrow M$ und $\cdot: M\times M \rightarrow M$, deren Elemente die Gesetze $(G1) - (G4)$, $(M1) - M4)$ und $(D)$ erfüllt, heißt Körper. \cite{HM12}
		
	\subsection{Komplexe Zahlen}
	\begin{align}
		\C = \{ z &= x + iy:\; x,\;y \in \R \}\\
		i^2 &= 1
	\end{align}
	
	Den Realteil einer komplexen Zahl $z$ bezeichnet man i.A. mit $x$, den Imaginärteil mit $y$.
	\begin{equation}
	  Re\{z\} = x \qquad Im\{z\} = y
	\end{equation}
	
	\defF \glqq Zwei komplexe Zahlen sind gleich, wenn ihre Real- und Imaginärteile gleich sind. \grqq \cite{HM12}
	
		\subsubsection{Betrag der komplexen Zahl}
		
		Da sich komplexe Zahlen geometrisch im $\R^2$ interpretieren lassen, kann aus beiden Teilen ein Betragspfeil gebildet werden.
		\begin{equation}
		  |z| = \sqrt{x^2 + y^2}
		\end{equation}
		
			\subsubsubsection{Eigenschaften}
			\begin{align}
				|z| &= 0\\
				|z| = 0 &\Leftrightarrow z = 0\\
				|z_1 \cdot z_2| &= |z_1| |z_2|\\
				|z_1 + z_2| &\leq |z_1| + |z_2| \quad \text{\colBlue{(Dreiecksungleichung)}}
			\end{align}
		
		\subsubsection{Komplex konjugierte Zahl}
		\defF \glqq $\overline{z} = x -iy$ heißt die zu $z = x + iy$ konjugiert komplexe Zahl. \grqq \cite{HM12}
		
			\subsubsubsection{Eigenschaften}
			\begin{align}
				\overline{\overline{z}} &= z \\
				\overline{z_1 + z_2} = \overline{z_1} + \overline{z_2},&\qquad \overline{z_1 \cdot z_2} = \overline{z_1} \cdot \overline{z_2} \\
				Re\{z\} = \frac{1}{2} /z + \overline{z}), &\qquad Im\{z\} = \frac{1}{2i} /z - \overline{z})\\
				|z| = \sqrt{z \overline{z}}, &\qquad z\overline{z} = x^2 + y^2
			\end{align}
			
		\subsubsection{Polarkoordinatendarstellung}
		\begin{align}
			z = x + iy &= |z|(cos\varphi + isin\varphi)\\
			\Rightarrow x = |z| cos\varphi, &\qquad \rightarrow y = |z| sin\varphi) \nonumber\\
			\Rightarrow tan \varphi &= \frac{sin \varphi}{cos \varphi} = \frac{x}{y}
		\end{align}
		
		Achtung. Die Umkehrfunktion von Tanges ist nicht eindeutig. Es gilt:
		
		\begin{align}
		\varphi = 
		\begin{cases} 
		    arctan(\frac{x}{y}) \qquad &,\; x > 0,\; y\geq 0\\
		    \frac{\pi}{2}  &,\;x = 0,\; y > 0\\
		    \pi + arctan(\frac{y}{x}) &,\; x < 0\\
		    \frac{3\pi}{2} &,\; x=0,\;y< 0\\
		    2\pi + arctan(\frac{y}{x}) &,\; x> 0,\; y < 0
		\end{cases}
		\end{align}
	
	  \subsubsection{Multiplikation}
	  \begin{align}
	  z_1 \cdot z_2 &\;\; = |z_1|(cos\varphi_1 + isin\varphi_1) \cdot |z_2| (cos\varphi_2 + isin\varphi_2)\\
	  &\overset{\eqref{eq:trigo_addtheo}}{=} |z_1| |z_2| (cos(\varphi_1 + \varphi_2) + isin(\varphi_1 + \varphi_2))\\
	  &\;\; \Rightarrow |z_1 \cdot z_2| = |z_1| |z_2| \qquad ,\; Winkel = \varphi_1 + \varphi_2 \label{eq:zahl_kompl_mult}
	  \end{align}
	  
	  \subsubsection{Formel von de Moivre}
	  Setzt man in \eqref{eq:zahl_kompl_mult} $z_1 = z_2 = z$ ein erhält man die Formel von de Moivre.
	  \begin{align}
	    z \cdot z &= z^2 = |z|^2 (cos(2\varphi) + isin(2\varphi)) &,\; bzw.&&\nonumber\\
	    z^n &= |z|^n(cos(n\varphi)  + isin(n \varphi)) &,\; bzw.&&\nonumber\\
	    z^n& = |z|^n(cos\varphi + isin\varphi)^n\nonumber\\
	    &\Rightarrow (cos\varphi + isin\varphi)^n = cos(n\varphi) + isin(n\varphi)
	  \end{align}
	
	  Mit der Eulerschen Formel erhält man $e^{i\varphi} := cos\varphi + isin\varphi$. Daraus folgt:
	  \begin{align}
	  (e^{i\varphi})^n &= e^{in\varphi} \\
	  e^{i(\varphi_1 + \varphi_2} &= e^{i\varphi_1} e^{i\varphi_2}\\
	  e^{-i\varphi} &= \frac{1}{e^{i\varphi}}\\
	  z_x = |z_x|e^{i\varphi_x} &\Rightarrow z_1 z_2 = |z_1| |z_2| e^{i(\varphi_1 + \varphi_2)}
	  \end{align}

  \subsection{Polynome}
  Ein komplexes Polynom hat die Form
  \begin{align}
    p(z) = \sum_{k=0}^n a_k z^k = a_n z^n + a_{n-1} z^{n-1} + ... + a_0 \\
    a_k \in \C \nonumber
  \end{align}
  Falls $a_n \neq 0$ gibt $n$ den Grad des Polynoms an.
  \newline
  \defF \glqq Ist $p(z)$ ein reelles Polynom, d.h. $a_k \in \R$, dann ist mit $z\in \C$ auch 
  $\overline{z} \in \C$ eine Nullstelle, d.h. aus $p(z) = 0$ folgt $p(\overline{z}) = 0$, d.h. die
  Nullstellen sind konjugiert komplex zueinander. \grqq \cite{HM12}

	\subsubsection{Fundamentalsatz der Algebra}
	  Jedes Polynom vom Grad $\geq$ 1 besitzt in $\C$ mindestens eine Nullstelle. 
  
  \subsubsection{Polynomdivision}
    Über Polynomdivision lassen sich Polynome in Linearfaktoren zerlege. 
    \begin{align}
      Bsp.:  \nonumber\\
      (2z^3 - 3z^2 -6z +6) : (z-2) = (2z^2 +z -4)\quad ,\text{ Rest: } -2 \nonumber\\
      \underset{\text{\rule{3cm}{0.4pt}}}{-(2z^3 - 4z^2)} \nonumber \\
      0 + z^2 - 6z + 6 \nonumber \\
      \qquad \underset{\text{\rule[5mm]{3cm}{0.4pt}}}{z^2-2z} \nonumber \\
    \end{align}








\section{\underline{Integralberechnung}}
\subsection{Unbestimmtes Integral}
\begin{equation}
\int f(x) dx = F(x) + C = [F(x)]\qquad, C\in\R \label{eq:def_noBorder}
\end{equation}

\subsection{Bestimmtes Integral}
\begin{equation}
\int_a^b f(x) dx = F(b) - F(a) \label{eq:def_border}
\end{equation}

\subsection{Partielle Integration}
Entspricht der "Produktregel" der Differentialrechnung.
\begin{equation}
\int_{\colBord{a}}^{\colBord{b}} f'(x) g(x) dx = f(x) g(x) \colBord{\Big|_a^b} - \int_{\colBord{a}}^{\colBord{b}} f(x) g'(x) dx \label{eq:rule_partInt}
\end{equation}
Bietet sich zum Beispiel bei Produkten aus x-Potenz mit e-Funktionen, log, sin oder cos an.

\subsection{Integration durch Substitution}
Entspricht der "Kettenregel" der Differentialrechnung.
\begin{equation}
\int_{\colBord{a}}^{\colBord{b}} f(g(x))g'(x) dx = \int_{\colBord{g(a)}}^{\colBord{g(b)}} f(y) dy \qquad (setze \quad y = g(x) \label{eq:rule_subs}
\end{equation}

\subsubsection{Spezialfall}
\begin{equation}
\int \frac{f'(x)}{f(x)} dx = ln(|f(x)|) + C \qquad ,C\in\R \label{eq:rule_spec}
\end{equation}

\newpage

\subsection{Gerade/Ungerade Funktionen}
\begin{align}
\int_{-a}^a f(x) = 
\begin{cases}
2 \int_0^a f(x) dx &,\; f\; gerade\\
0 &,\; f\; ungerade\\
\end{cases} \label{eq:evenodd}
\end{align}
\begin{align*}
\text{f gerade, falls }f(-x) &= f(x) \qquad &(z.B.: cos(x), x^2)\\
\text{f ungerade, falls }f(-x) &= -f(x) \qquad &(z.B.: sin(x), x^3)\\
\end{align*}

\subsection{Beispiele}
Siehe Anhang

\subsection{Allgemeines zur Integration}
\subsubsection{Riemann Integrierbarkeit}
$f:[a,b] \rightarrow \R$ stetig bzw. monoton \newline
$\Rightarrow$ f ist R-integrierbar.

\subsubsubsection{Riemannsches Unterintegral}
\begin{equation}
\int_{a}^{\bar{b}} f(x) dx = \sup\{U_f(Z): \; \text{Z Zerlegung von }[a,b]\}
\end{equation}


\subsubsubsection{Riemannsches Oberintegral}
\begin{equation}
\int_{\bar{a}}^{b} f(x) dx = \inf\{O_f(Z): \; \text{Z Zerlegung von }[a,b]\}
\end{equation}

$\rightarrow \text{f heißt Riemann-integrierbar über }[a,b]$, falls
\begin{equation}
\int_{\bar{a}}^{b} f(x) dx = \int_a^{\bar{b}} f(x) dx
\end{equation}
\newline
In diesem Fall heißt der Wert das Riemannn-Integral und wird mit $\int_a^b f(x)dx$ bezeichnet.

\subsubsubsection{Eigenschaften}
\begin{description}
\item[a)]
Falls $a<b$ setzen wir:
\begin{align}
\int_b^a f(x) dx &= -\int_a^bf(x)dx \nonumber \\
\int_a^a f(x) dx &= 0
\end{align}
\item[b)]
f, g seien R-integrierbar, $\lambda , \mu \in \R \rightarrow \lambda f + \mu g$ ist R-integrierbar (Vektorraumeigenschaft).
\begin{equation}
\int_a^b \lambda f + \mu g)(x)dx = \lambda \int_a^b f(x) dx + \mu \int_a^b g(x) dx
\end{equation}
\item[c)]
$a<C<b$, f ist R-integrierbar.
\begin{equation}
\int_a^b f(x) dx = \int_a^C f(x) dx + \int_C^b f(x) dx
\end{equation}
\item[d)]
\begin{align}
f(x) \ge 0 &\Rightarrow \int_a^b f(x) dx \ge 0 \nonumber \\
f(x) \ge g(x) &\Rightarrow \int_a^b f(x) dx \ge \int_a^b g(x)dx
\end{align}
\item[e)]
\begin{equation}
\text{Sind $f$ und $g$ R-integrierbar ist auch $f*g$ R-integrierbar.}
\end{equation}
\item[f)]
\begin{align}
g(x) \ge C > 0 \Rightarrow \frac{f}{g} \text{ ist R-integrierbar.}
\end{align}
\item[g)]
\begin{equation}
\text{Ist $f$ R-integrierbar dann ist auch } |f| \text{ R-integrierbar.}
\end{equation} 
\item[h)]
\begin{equation}
(b-a) \inf_{x\in[a,b]}{f(x)} \le \int_a^b f(x) dx \le (b-a) \sup_{x\in [a,b]}{f(x)}
\end{equation}
\end{description}

\subsubsubsection{Kriterien zur Riemann-Integrierbarkeit}

\begin{description}
\item[a)]
$f$ monoton $\Rightarrow f$ R-integrierbar.
\item[b)]
$f$ stetig $\Rightarrow f$ R-integrierbar
\newline
\glqq Satz: Jede stetige Funktion $f:k \rightarrow \R$ auf einer kompakten Menge k, d.h. für $k<\R^d$ abgeschlossen und beschränkt, ist dort gleichmäßig stetig und damit R-integrierbar.\grqq \cite{HM12}
Beispiel für k: $k:[a,b]$
\item[c)]
Kriterium: Jede Funktion deren Unstetigkeitsstellen eine Nullmenge bilden (z.B. abzählbare Mengen) sind R-integrierbar.
\glqq Satz: Eine Funktion $f:[a,b]\rightarrow \R$ ist genau dann R-integrierbar, wenn $f$ beschränkt ist und die Menge der Unstetigkeitsstellen eine Nullmenge ist. \grqq \cite{HM12}
Die Konsequenz daraus lautet, dass jede stetige Funktion mit endlich vielen Sprungstellen R-integrierbar ist. \citeVgl{HM12}
\item[d)]
\glqq Satz: Sei $f:[a,b] \rightarrow \R$ beschränkt. Dann ist $f$ R-integrierbar genau dann, wenn es zu jedem $\varepsilon > 0$ eine Partition $Z$ gibt, 
so dass
$O_f(Z)  U_f(Z) < \varepsilon$. \grqq \cite{HM12}
\newline
Anmerkung: \glqq In der Mengenlehre ist eine Partition (auch Zerlegung oder Klasseneinteilung) einer Menge M eine Menge P, deren Elemente nichtleere Teilmengen von M sind, sodass jedes Element von M in genau einem Element von P enthalten ist. Anders gesagt: Eine Partition einer Menge ist eine Zerlegung dieser Menge in nichtleere paarweise disjunkte Teilmengen.\grqq  \cite{wiki}

\end{description}

\subsubsection{MWS der Integralrechnung}
$f:[a,b]\rightarrow\R$ stetig, dann $\exists \; \xi \in[a,b]$ mit $\int_a^b f(x)dx = f(\xi)(b-a)$.

\subsubsection{Hauptsatz der Differential- und Integralrechnung}
$f:[a,b]\rightarrow\R$ stetig, dann ist $F(x) = \int_a^x f(t)dt$ diffbar und $F'(x) = f(x)$.

\subsubsection{Anwendungen}
\begin{description}
\item[1)]
\begin{align}
\lim_{x\to 0} \frac{1}{x} \int_0^x e^{-cos(y^{17})} dy = \lim_{x\to 0}  \frac{\overbrace{\int_0^x e^{-cos(y^{17})}}^{\colBord{\rightarrow 0}}}{\underbrace{x}_{\colBord{\rightarrow 0}}} \nonumber\\
\overset{\colBord{"\frac{0}{0}" \rightarrow L.H.}}{=}
\lim_{x \to 0} \frac{e^{-cos(x^{17}}}{1} = e^{-1} = \frac{1}{e}
\end{align}
\item[2)]
\begin{align}
\lim_{x \to \infty} xe^{x^2} \int_0^x e^{y^2}dy 
&= \lim_{x \to \infty} \frac{\overbrace{\int_0^x e^{y^2}dy}^{\colBord{\rightarrow \infty}}}
{\underbrace{\frac{1}{x} e^{x^2}}_{\colBord{\rightarrow \infty}}} \nonumber\\
\overset{\colBord{"\frac{\infty}{\infty}" \rightarrow L.H.}}{=}
\lim_{x \to \infty} \frac{e^{x^2}}{-\frac{1}{x^2}* \colRed{\cancel{e^{x^2}}}
+ \frac{1}{x} 2x \colRed{\cancel{e^{x^2}}}} 
&= \lim_{x \to \infty} \frac{1}{- \frac{1}{x^2} +2} = \frac{1}{2}
\end{align}
\end{description}
\colRed{Add rest of integrals here!}
\newpage

\section{Separierbare DGL}
  \subsection{Wiederholung klassische DGL}
  Bisher: lineare DGl mit konstanten Koeffizienten. \newline
  z.B.: $y''(t) - 5y'(t) + 4y(t) = e^{\colGreen{2}t} \qquad ,\; y(0) = 1,\; y'(0) = 1$ \newline
  \begin{tabularx}{14.7cm}{l l}
	  Homogene DGL: & $y(t) = e^{\lambda t} \Rightarrow p(\lambda) = \lambda^2 - 5 \lambda +4 = 0$ \\
	  $\;$ & $\Rightarrow \lambda_1 = 1, \; \lambda_2 = 4$ \\
	  $\;$ & $\Rightarrow yh(t) = C_1 e^t + C_2 e^{4t} \qquad ,\;C_1,\;C_2 \in \R$\\
	  $\;$ & $\;$ \\
	  Inhomogenes DGL: & $\ubGreen{\text{da 2 keine NST}}{yp(t) = re^{2t}}$\\
  \end{tabularx}  
  \begin{align*}
    &\Rightarrow yp'(t) = 2re^{2t},\; yp''(t) = 4re^{2t} \\
    &\overset{DGL}{=} 4re^{2t} - 10re^{2t} + 4re^{2t} \overset{!}{=} e^{2t} \Rightarrow -2re^{2t} = e^{2t}\\
    &\Rightarrow r = -\frac{1}{2}
  \end{align*} 
  \begin{tabularx}{14.7cm}{l l}
	  Allgemeine Lösung: & $y(t) = yh(t) + yp(t) = C_1 e^t + C_2 e^{4t} - \frac{1}{2} e^{2t}$
  \end{tabularx}
  
  \subsection{Lösen von DGL mit Koeffizienten die von t abhängig sind}
  z.B. $ y'(t) - ty(t) = t \qquad ,\; y(0) = 1$\newline
  \newline
  Spezielle Form: 
  \begin{equation}
    y'(t) = f(t) g(y(t)) \qquad,\; y(t_0) = y_0
  \end{equation}     
  \begin{align*}
    \Rightarrow y'(t) = t+ty(t) = \ubGreen{f(t)}{t} \ubGreen{(1+y(t))}{g(y(t))}
  \end{align*}
  Lösung: Trennung der Veränderlichen:
  \begin{align}
  \frac{y'}{g(y)} = f(t) \overset{\colGreen{y' = \frac{dy}{dt}}}{\Rightarrow} \colGreen{\int} \frac{1}{g(y)} dy = \colGreen{\int} f(t)dt +C \quad ,\;C\in\R
  \end{align}
  $C$ erhält man aus der Anfangsbedingung $y(t_0) = y_0$.
  \subsubsection{Beispiele}
  Siehe Anhang
	
\section{Lineare Algebra}
  \subsection{Definitionen}
  \subsubsection{Linearkombinationen}
  $V$ sei ein Vektorraum (im Folgenden VR), $v_1,\;...,\;v_m \in V\quad ,\;\lambda  _i \in \R$.
  \begin{equation}
		  \begin{tabularx}{14.7cm}{l l l}
				(\textbf{1}) & $\sum\limits_{i = 1}^m \lambda_i v_i$ & \colBlue{Linearkombinationen der $v_i$.}\\
		  \end{tabularx}
		  \label{def:linA_lineark}
    \end{equation}
    \subsubsection{Spann und Erzeugendensystem}
    \begin{equation}
		  \begin{tabularx}{14.7cm}{l l l}
				(\textbf{2}) & $span(v_1,\;...,\;v_m) = \left\lbrace \sum\limits_{i = 1}^m \lambda_i v_i \quad ,\; \lambda_i,\; v_i \in \R \right\rbrace$ & \colBlue{Spann der
				 $v_i$.}\\
				 $\;$ & Gilt $span(v_1,\;...,\;v_m) = V \Rightarrow \lbrace v_1,\;...,\;v_m \rbrace$ &ist ein Erzeugendensystem.\\
		  \end{tabularx}
		  \label{def:linA_span}
    \end{equation}
    \subsubsection{Lineare Unabhängigkeit}
    \begin{equation}
		  \begin{tabularx}{14.7cm}{l l l}
				(\textbf{3}) & \multiTwo{l}{$v_1,\;...,\;v_m$ linear unabhänig, falls}\\ 
				$\;$ & \multiTwo{l}{$\sum\limits_{i = 1}^m \lambda_i v_i = \vec{0} \Rightarrow \lambda_1 = \lambda_2 = ... = \lambda_m = 0$}\\
				$\;$ & \multiTwo{l}{$0$ darf die einzige Lösung sein, sonst linear abhängig.}\\
		  \end{tabularx}
		  \label{def:linA_linearabh}
    \end{equation}
    \subsubsection{Basis}
    \begin{equation}
		  \begin{tabularx}{14.7cm}{l l l}
				(\textbf{4}) & \multiTwo{l}{$B = \lbrace b_1,\;...,\;b_n \rbrace \subset V$ ist Basis von $V$, falls}\\
				$\;$ & \colGreen{(B1)} & $b_i$ linear unabhängig, $i = 1,\; ...,\; n$\\
				$\;$ & \colGreen{(B2)} & $B$ ist ein Erzeugendensystem.\\
		  \end{tabularx}
		  \label{def:linA_basis}
    \end{equation}	
    
    Es gilt:
    \begin{equation}
      \begin{tabularx}{14.7cm}{l l l}
      (\textbf{1}) & $dimV = |b|$ & \colBlue{(Mächtigkeit von $B$)}\\
      (\textbf{2}) & $dimV = n$ & \colBlue{($\Rightarrow n+1$ Vektoren sind linear abhängig)}\\
      (\textbf{3}) & \multiTwo{l}{$\forall v \in V: v = \sum\limits_{i = 1}^n \lambda_i b_i$ eindeutig darstellbar.}\\
      $\;$ & $\Rightarrow B^v = (\lambda_1,\;...,\; \lambda_n)^T$ & \colBlue{(Koordinaten von $v$ bezüglich $B$)}\\
      \end{tabularx}
      \label{def:linA_allg}
    \end{equation}

  \subsubsection{Kanonische Basis}
  \begin{equation}
    e_1 = \vecT{1\\0\\ \dddot \\ 0,} \quad e_2 = \vecT{0\\1\\ \dddot \\ 0}, \quad ..., \quad e_d = \vecT{0\\ \dddot \\0 \\1}
  \end{equation}
	\subsection{Vektorräume}
		\subsubsection{Definitionen}
		\subsubsubsection{$\R^d$}
		\begin{equation}
			\R^d: v = \vecT{v_1\\v_2\\ \dddot \\v_d},\; w = \vecT{w_1\\w_2\\ \dddot \\ w_d}
		\end{equation}	
		\subsubsubsection{Vektoraddition}
		\begin{equation}
			v + w = \vecD{v} + \vecD{w} = \vecT{v_1 + w_1 \\ \dddot \\ v_d + w_d}
		\end{equation}
		\subsubsubsection{Skalare Multiplikation}
		\begin{align}
			\alpha \in \R, \; v \in \R^d \nonumber \\
			\alpha \cdot v = \vecD{\alpha\cdot v}
		\end{align}
		\subsubsubsection{Nullvektor}
		\begin{equation}
		  \mathcal{O} = \vecT{0 \\ \dddot \\ 0}
    \end{equation}		
    
    \subsubsection{Struktur}
    Für $v,\;w,\;z \in \R^d$ gelten die folgenden Eigenschaften:
    \begin{equation}
		  \begin{tabularx}{14.7cm}{l l l}
				(\textbf{V1}) & $v + w = w + v$ &\\
				(\textbf{V2}) & $v+(w+z) = (v+w)+z$ &\\
				(\textbf{V3}) & $v + 0 = 0 + v = v$ & \colBlue{(0 ist das neutrale Element)}\\
				(\textbf{V4}) & $v+(-v) = (-v)+v = 0$ & \colBlue{(-v ist das inverse Element)}\\
		  \end{tabularx}
		  \label{ax:vektorraumeig_v}
    \end{equation}	
    \newline
    Für $\alpha\; \beta \in \R,\; v,\;w \in \R^d$ gilt:
    \begin{equation}
		  \begin{tabularx}{14.7cm}{l l l}
				(\textbf{S1}) & $1 \cdot v = v$ & $\qquad\colBlue{(1\in \R)}$\\
				(\textbf{S2}) & $\alpha (\beta v) = (\alpha \beta)v$ &\\
				(\textbf{S3}) & $(\alpha + \beta) v = \alpha \cdot v + \beta \cdot v$ &\\
				(\textbf{S4}) & $\alpha(v + w) = \alpha \cdot v + \alpha \cdot w$ &\\
		  \end{tabularx}
		  \label{ax:vektorraumeig_s}
    \end{equation}		 
    $(V1) - (V4)$ und $(S1) - (S4)$ gelten auch für Funktionen.
    \begin{definition}
    Ist $V$ eine Menge für deren Elemente eine Addition und eine skalare Multiplikation erklärt ist, so heißt sie Vektorraum, falls die Eigenschaften 
    $(V1)$ - $  (V4)$ und $(S1)$ - $(S4)$ erfüllt sind, wobei jetzt $v,\;w,\;z\in V$.
    Je nach Skalarkörper, also $\R$ oder $\C$ sprechen wir von einem reellen oder komplexen Vektorraum. Teilmengen von Vektorräumen, die ebenfalls Vektorräume
    sind, heißen Untervektorräume. \cite{HM12}
    \end{definition}

	  \subsubsection{Untervektorraum}
	  $V$ sei ein Vektorraum und es gelte $U \subset V$.
	  \begin{definition}
	  Eigenschaften $(V1)$ - $(V4)$, $(S1)$ - $(S4)$ sind als Teilmenge von V erfüllt, aber mit $u,\;v \in U$ und $\alpha \in \R$ muss auch $u + v \in U$
	  , $\alpha  u \in U$ gelten (Abgeschlossenheit bezüglich Vektoraddition und skalarer Multiplikation). \cite{HM12}
	  \end{definition}
	  \subsubsubsection{Untervektorraumkriterien}
	  \begin{equation}
			  \begin{tabularx}{14.7cm}{l l l}
					(\textbf{UV0}) & $0 \in U$ &\\
					(\textbf{UV1}) & $u,\;v \in U \Rightarrow u + v \in U$ &\\
					(\textbf{UV2}) & $u \in U,\; \lambda \in \R \Rightarrow \lambda u \in U$ &\\
			  \end{tabularx}
			  \label{ax:vektorraumeig_uv}
	   \end{equation}
	   \newline
	   Es gilt: $U_1,\; U_2 \;$ UVR von V
	   \begin{equation}
	     \begin{tabularx}{14.7cm}{l l l}
					(\textbf{1}) & $U_1 \cap U_2 = \{v\in V: v \in U_1 \land v \in U_2 \}$ UVR von  V &\\
					(\textbf{2}) & $U_1 \cup U_2 = \{v\in V: v \in U_1 \lor v \in U_2\}$ kein UVR von V &\\
			  \end{tabularx}
			  \label{ax:vektorraumeig_uv}
	   \end{equation}
	   \subsubsubsection{Triviale UVR von V}
	   \begin{align*}
		   U = \{0\}\\
		   U = V
	  \end{align*}
	 
	  \subsubsubsection{Interessante Untervektorräume}
	  \begin{itemize}
	    \item Die Menge der stetigen Funktionen: $\brac{\brac{\bracks{a,b},\R}}$
	    \item Die Menge der n-mal stetig diffbaren Funktionen $\C^n \brac{\bracks{a,b} , \R}$
	    \item Die Menge der Riemann-integrierbaren Funktionen
	  \end{itemize}
	  
	  \subsection{Erzeugendensystem, Basis, Dimension, lineare Unabhängigkeit}
	  \subsubsection{Linearkombination, Spann}
	  \begin{definition}
	    Für $v_1,...,v_m \in V$(Vektorraum) und $\lambda_j \inRs $ heißt ein Vektor der Form 
	    \begin{equation}
	      v = \sum_{j = 1}^m \lambda_j v_i
	    \end{equation}
	    eine Linearkombination der Vektoren $v_1,...,v_m$. Die Menge aller Linearkombinationen heißt der Spann von $v_1,...,v_m$, d.h.
	    \begin{equation}
	      Spann(v_1,...,v_m) = \lbrace \sum_{j=1}^m \lambda_j v_j: \lambda_j \inRs \rbrace
      \end{equation}	     
      bzw. der von $v_1,...,v_m$ aufgespannte Raum. \cite{HM12}
	  \end{definition}
	  
	  Beispiel:
	  \begin{align*}
	    v_1 = \vecT{1\\0\\1}, \quad v_2 = \vecT{2\\1\\1}, \quad v_3 &= \vecT{3\\1\\2}, \quad v_4 = \vecT{1\\1\\0}\\
	    v_3 = v_1 + v_2 \quad \quad v_4 &= v_2-v_1\\
	    \Rightarrow Spann\brac{v_1,...,v_4} &= \lbrace \lambda_1 v_1 + \lambda_2 v_2: \lambda_1, \lambda_2 \inRs \rbrace
	  \end{align*}
	  
	  \subsubsection{Lineare Unabhängigkeit}
	  \begin{definition}
	    Die Vektoren $v_1,...,vm$ heißen linear unabhängig, falls aus 
	    \begin{equation}
	      \sum_{j=1}^m \lambda_j v_j = \vecN \in V
	    \end{equation}
	    bereits 
	    \begin{equation}
	      \lambda_1 = ... = \lambda_m = 0 \inRs 
	    \end{equation}
	    folgt.
	  \end{definition}
	  
	  \subsubsection{Basis}
	  \begin{definition}
	    $B = \lbrace v_1,...,v_M\rbrace \subset V$ heißt eine Basis von $V$, falls $B$ linear unabhängig und $V = spann \lbrace v_1,...,v_m \rbrace$ ist.
	  \end{definition}
	 
    \begin{satz}
      \glqq (Basen von endlich-dimensionalen Vektorräumen sind gleich groß).\newline
      Sei $V$ ein Vektorraum mit Basis $\lbrace v_1,...,v_m\rbrace$. Dann sind je $n$ Vektoren $w_1,...,w_n$ aus $V$ mit $n>m$ linear abhängig. \grqq \cite{HM12}
    \end{satz}
     
     \subsubsubsection{Folgerungen}
     \begin{definition}
		  \glqq Ist $\lbrace v_1,...,v_m \rbrace$ eine Basis des Vektorraums $V$, so lässt sich jeder Vektor $v \in V$ eindeutig als Linearkombination der $\lbrace               
		    v_1,...,v_m$ schreiben, d.h.:
		    \begin{equation}
		      \exists x_k \text{ mit } v = \sum_{k=1}^m x_k v_k
		    \end{equation}
		    \grqq \cite{HM12}\newline
		    Mit $x_k$ und $v_k$ als Koordinaten bezüglich dieser Basis.
	    \end{definition}
	    
	    \begin{definition}
		    \glqq Die Anzahl der Elemente einer Basis von $V$ ist unabhängig von der speziellen Wahl der Basis. Die Anzahll der Elemente der Basis heißt die Dimension des Vektorraumes $V$.\grqq \cite{HM12} Schreibweise:
		    \begin{equation}
		      dim V = m
		    \end{equation}
		   \end{definition}
		   
		   \mDef{
		     \glqq Besitzt $V$ eine Basis mit endlich vielen Elementen, so heißt $V$ edlich dimensional, sonst heißt $V$ unendlich dimensional. \grqq \cite{HM12}
		   }
	    
	    \subsection{Lösungsmengen linearer Gleichungssysteme}
	    \begin{align}
		    \left(
	      \begin{array}{c c c c c c c}
	      a_{11}x_1 & + &  ...   & + & a_{1n}x_n & =  & b_1\\
	      .         & \;& \;     & \;& .         & \; & . \\
	      .         & \;& \;     & \;& .         & \; & . \\
	      .         & \;& \;     & \;& .         & \; & . \\
	      a_{m1}x_1 & + & ...    & + & a_{mn}x_n & =  & b_m\\
	      \end{array}
	    \right) \rightarrow Ax = b   
	   \end{align} 
	   
	   \subsubsection{Zeilenvektoren}
	   Die Koeffizienten der Zeilen nennt man Zeilenvektoren.
	   \begin{equation}
	     v_1 = \vecDt{a_{11}}{a_{1n}}, \quad ..., \quad v_m = \vecDt{a_{m1}}{a_{mn}}
	   \end{equation}
	   
	   \subsubsection{Gauss Algorithmus}
	   Beim Gauss-Algorithmus werden Linearkombinationen der Zeilenvektoren gebildet. Das LGS hat nach der Anwendung folgende Form:
	   
	   \begin{align}
	     \begin{array}{c c c c c c c c c c l}
	       \tilde{a}_{11} x_1 & +   & ...                & + & \tilde{a}_{1r} x_r & + & ...                & + & \tilde{a}_{1n} x_n   & = & \tilde{b}_1\\
	       0                  & +   & \tilde{a}_{22} x_1 & + & ...                & + & \tilde{a}_{2r} x_r & + & ...                  & = & \tilde{b}_1\\
	       \;                 & .   & \;                 &\; & \;                 &\; & \;                 &\; & \;                   &\; & \;         \\
	       \;                 &\; . & \;                 &\; & \;                 &\; & \;                 &\; & \;                   &\; & \;         \\
	       \;                 &\;   & .                  &\; & \;                 &\; & \;                 &\; & \;                   &\; & \;         \\
	       \;                 &\;   & \;.                &\; & \;                 &\; & \;                 &\; & \;                   &\; & \;         \\
	       \;                 &\;   & \;                 &\, & \tilde{a}_{rr} x_r & + & ...                & + & \tilde{a}_{rn} xn    & = & \tilde{b}_r\\
	       \;                 &\;   & \;                 &\; & \;                 &\; & \;                 &\; & 0                    & = & \tilde{b}_r + 1\\ 
	       \;                 &\;   & \;                 &\; & \;                 &\; & \;                 &\; & 0                    & = & \tilde{b}_n\\
	     \end{array}
	   \end{align}
	   Die neuen Zeilenvektoren nach dem Gauss-Algorithmus sind:
	   \begin{equation}
	     \tilde{v_1} = \vecDt{\tilde{a}_{11}}{\tilde{a}_{1n}}, \quad ..., \quad \tilde{v_m} = \vecDt{\tilde{a}_{m1}}{\tilde{a}_{mn}}
	   \end{equation}
	   
	   Da beim Gauss-Algorithmus nur Linearkombinationen verwendet werden bleibt der Spannn erhalten.
	   
	   \subsubsection{Gauss-Algorithmus: Spann}
	   \begin{equation}
	     \vspan{v_1, ..., v_m} = \vspan{\tilde{v}_1,...,\tilde{v}_m}
	   \end{equation}
	   
	   \subsubsection{Gauss-Algorithmus: Dimension, Rang}
	   Weiterhin gilt:
	   
	   \begin{equation}
	   dim \lbrace \vspan{v_1,...,v_m} \rbrace = dim \lbrace \tilde{v}_1,...,\tilde{v}_m \rbrace = r
	   \end{equation}
	   $r$ heißt Zeilenrang von $A$ bzw. der Rang von $A$.
	   
	   \subsubsection{Folgerungen}
	   \begin{satz}$\\$
	   \textbf{a)}
	     \glqq Die Lösungen von $Ax = 0$ bilden einen Untervektorraum des $\R^n$, den sogenannten Kern von $A$. \grqq \cite{HM12} \newline
	     Schreibweise:
	     \begin{equation}
	       Kern(A) = ker(A) = kernel(A)
	     \end{equation}
	     \newline
	     \textbf{b)}
	     \begin{equation}
	       \ubGreen{n-r}{\vdim{\vker{A}}} + \ubGreen{r}{\vrang{A}} = n
	     \end{equation}
	     \newline
	     \textbf{c)}
	     \ccite{Ein LGS $Ax = 0$ mit $A \in \R^{nxn}$ besitzt nur die Lösung $x = 0$, wenn $\\\vrang{A} = n$.}{HM12}
	     \newline
	     \textbf{d)}
	     \ccite{Ist $\left(w_1,...,w_k\right)$ eine Basis des Kerns, so lautet die allgemeine Lösung von $Ax = 0$:
	     \begin{equation}
	       x = sum_{j = 1}^k \alpha_j w_j \qquad ,\alpha_j \in \R \qquad \colGreen{(Superpositionsprinzip)}
	     \end{equation}}{HM12}
	     \newline
	     \textbf{e)}
	     \ccite{Ist $x_s$ eine spezielle Lösung von $Ax = b$, so lautet die allgemeine Lösung von $Ax = b$:
	     \begin{equation}
	       x = x_s + \sum_{j = 1}^k \alpha_j w_j \qquad, \inR{\alpha_j}
	     \end{equation}
       Die Lösungsmenge von $Ax = b$ ist ein \underline{affiner Raum}, d.h. die Differenz von jeweils zwei Elementen bildet einen Vektorraum.}{HM12}
	   \end{satz}
	   
  \subsection{Lineare Abbildungen und Matrizen}
  \begin{definition}
    \ccite{Es seien $V$ und $V$ Vektorräume. Eine Abbildung $T: V\rightarrow W$ heißt linear, falls für alle $u,v \in W$ und $\lambda \in \R$ bzw. $\C$
    \begin{align}
      T(v+w) &= T(v) + T(w)\\
      T(\lambda v) &= \lambda T(v)
    \end{align}
    gilt.}{HM12}
  \end{definition}
  
  Wie erhält man die Matrix zu einer linearen Abbildung?\newline
  \ccite{
    Sei $T:V \rightarrow W$ eine lineare Abbildung. $\lbrace v_1,...,v_n\rbrace$ sei eine Basis von $V$ und $\lbrace w_1,...,w_n\rbrace$ sei eine Basis von W. Dann lässt sich jeder Vektor $T(v_j)$ in eindeutiger Weise als Linearkombintion der $w_1,...,w_m$ darstellen, d.h. es gibt $a_{ij} \in \R$ mit 
    \begin{equation}
      T(v_j) = \sum_{i = 1}^m a_{ij} w_i\text{.}
    \end{equation}
    $
      A = \left( 
      \begin{array}{c c c}
        a_{11} & ... & a_{1n}\\
        .      & .   &.      \\
        a_{m1} & ... & a_{mn}\\
      \end{array}
      \right)
    $
    heißt die Matrix der linearen Abbildung $T$ bezüglich der Basen $\lbrace v_1,...,v_m\rbrace$ und $\lbrace w_1,...,w_m\rbrace$.
  }{HM12}
  
  \begin{satz}
  \ccite{
    Die Koordinaten von $w = T(v)$ entstehhen aus den Koordinaten von $v$ durch Multiplikation mit der Matrix $A$.  
    \begin{align}
      w &= T(v) \nonumber\\ 
      mit \; v &= \sum_{j = 1}^n x_j v_j \; und \; w = \sum_{i = 1}^m y_i w_i \nonumber\\
      \text{ergibt sich} \nonumber\\
      \vecDt{y_1}{y_m} &= 
      \left(\begin{array}{c c c}
        a_{11} & ... & a_{1n} \\
        .      & \;  & .      \\
        .      & \;  & .      \\
        .      & \;  & .      \\
        a_{m1} & ... & a_{mn} \\
      \end{array}\right) 
      \vecDt{x_1}{x_n}
    \end{align}
    bzw. $y = Ax$.
  }{HM12}
  \end{satz}
  
  \subsubsection{Das Matrizenkalkül}
  Matrizenmultiplikation tritt bei der Hintereinanderausführung linearer Abbildungen auf.
  \begin{equation}
    T(v) = Av,\quad S(w) = Bw \rightarrow (SoT)(v) = \ubGreen{Matrizen}{BA} \ubGreen{Vektoren}{v}
  \end{equation}
  Der Eintrag in der i-ten Zeile und j-ten Spalte von $BA$ sieht wie folgt aus:
  \begin{equation}
    a_{ij} = \sum_k b_{ik} a_{kj}
  \end{equation}
  
  Damit Matrizen multipliziert werden könnnen, müssen sie die richtige Größe haben.
  \begin{equation}
    Mit  \; A \in \R^{\colBlue{m}x\colGreen{n}} \; und \; B \in \R^{\colGreen{n}x\colBlue{k}} = C \in \R^{\colBlue{m}x\colBlue{k}}
  \end{equation}
  Matrizenmultiplikation ist im allgemeinen
  \begin{itemize}
	  \item nicht kommutativ, d.h. $AB \neq BA$ 
	  \item nicht nullteilerfrei
  \end{itemize}
  
  \subsubsection{Rechenregeln}
  \begin{equation}
		  \begin{tabularx}{14.7cm}{l l l}
				(\textbf{M1}) & $(A+B)C = AC + BC$ &\\
				(\textbf{M2}) & $A(B+C) = AB + AC$ &\\
				(\textbf{M3}) & $A(BC) = (AB)C$ &\\
				(\textbf{M4}) & $A(\lambda B) = (\lambda A)B = \lambda(AB)$ &\\
		  \end{tabularx}
		  \label{ax:matrizen_rregel}
    \end{equation}	
  \subsubsection{Einheitsmatrix}
  Die Einheitsmatrix (Identität) Besteht nur aus in der Diagonale Einsen, alle anderen Stellen sind mit Nullen aufgefüllt.
  \begin{equation}
    I = id = \left(\begin{array}{c c c c c}
    1 & \; &  0 & \; & 0\\
    \;& .  & \; & \; & 0\\
    0 & \; &  . & \; & 0\\
    \;& \; & \; & .  & 0\\
    0 & \;  & 0 & \; & 1\\
    \end{array}\right)
  \end{equation}
  
  Die Einträge an i-ter Zeile und j-ter Spalte sind über das Kronecker-Delta beschrieben.
  \begin{equation}
    \delta_{ij} = \begin{cases}
	    1,\quad i = j\\
	    0,\quad i\neq j\\
    \end{cases}
  \end{equation}
 
\newpage

\section{\underline{Anhänge}}
	\subsection{Ausgewählte Beispiele}
	  \subsubsection{Lineare Algebra}
	    \subsubsubsection{Vektor- und Untervektorräume}
	    \begin{align}
	      P_2 &= \lbrace p(x) = ax^2 + bx + c \quad ,\;a,\;b,\;c \in \R \rbrace \nonumber\\
	      B &= \lbrace  \ubGreen{b_1}{1},\; \ubGreen{b_2}{1-x} ,\; \ubGreen{b_3}{x^2-x} \rbrace \qquad Basis? \nonumber \\
	      \colGreen{(B_1)}&:\; \lambda_1 b_1 + \lambda_2 b_2 + \lambda_3 b_3 = \lambda_1 \cdot 1 + \lambda_2 (1-x) + \lambda_3 (x^2 -x) = 0 \quad ,\forall x \in \R
	      \nonumber \\
	      &\Rightarrow \ubGreen{=0}{\lambda_1 + \lambda_2} + \ubGreen{=0}{(-\lambda_2 - \lambda_3)}x+\ubGreen{=0}{\lambda_3}x^2 = 0 \quad ,\forall x \in \R
	      \nonumber \\
	      &\Rightarrow \lambda_1 = \lambda_2 = \lambda_3 = 0 \Rightarrow b_i \; \text{sind linear unabhängig} \nonumber \\
	      &\colGreen{\text{Alternativ: 3 Zahlen für x einsetzen und LGS lösen.}}
	      \\
	      \colGreen{(B_2)}&: dim P_2 = 3
	    \end{align}
		  \begin{description}
		    \item[\textbf{1)}]
		    \begin{align*}
		      U &= \lbrace x \in \R^2 : x = t \vecT{1\\-1},\; t \in \R \rbrace \text{,, UVR von } V = R^2? \\
		      &U \text{ ist Ursprungsgreade} \\
		      \colGreen{(UV0): \checkmark} &da \; \vecT{0\\0} \in U (\text{setze t} = 0)\\
		      \colGreen{(UV1): \checkmark} &da \;u,\;v \in U \Rightarrow \exists t_1,\; t2 \in \R \\
		      u& = t_1 \vecT{1\\-1}, \quad v = t_2 \vecT{1\\-1}\\
		      &\Rightarrow u + v = t_1 \vecT{1\\-1} + t_2 \vecT{1\\-1} = \ubGreen{r \in \R}{t_1 + t_2} \vecT{1\\-1} \in U\\
		      \colGreen{(UV2): \checkmark} &da u \in U \Rightarrow \exists t \in \R, \quad u = t \vecT{1\\-1}\quad ,\lambda \in \R \\
		      &\Rightarrow \lambda u = \ubGreen{\tilde{r} \in \R}{\lambda t} \vecT{1\\-1} \in U \\
		      &\Rightarrow u \text{ ist }UVR
		    \end{align*}
		    \item[\textbf{2)}]
		    \begin{align*}
		      U &= \lbrace x \in \R^2: x = \vecT{0\\1} + t \vecT{1\\-1},\;t\in\R \rbrace ,\;UVR \text{ von } \R?\\
		      \colGreen{(UV0): }\; \colRed{\lightning}&,\; \vecT{0\\0} = \vecT{1\\0} + t \vecT{1\\-1} \Rightarrow t = 0 \Rightarrow 0 = 1 \quad \colRed{\lightning} \\
		    \end{align*}
		    \item[\textbf{3)}]
		    \begin{align*}
		      U &= \lbrace (x,y) \in \R^2 : y = x^2 \rbrace UVR \text{ vom } \R^2? \\
		      \colGreen{(UV0): \checkmark} &(0,0) \in U,\;da\; 0 = 0^2\\
		      \colGreen{(UV2): }\; \colRed{\lightning} \;&(2,4) \in U\; da\; 4 = 2^2\\
		      &\lambda \in \R = 3 \Rightarrow 3(2,4) = (6,12) \notin U \; da \; 12 \neq 6^2 \Rightarrow kein \; UVR
		    \end{align*}
		    \item[\textbf{4)}]
		    \begin{align*}
		      U_1 &= \lbrace x \in \R^2: x =  r \vecT{1\\0},\;r \in \R \rbrace , \quad U_2 = \lbrace x \in \R : x = s \vecT{0\\1},\; s \in \R \rbrace \\
		      &U_1,\; U_2\; UVR \; von \; \R^2\\
		      &\text{a) } U_1 \cap U_2 = \lbrace (0,0) \rbrace \; UVR \; von \R^2\\
		      &\text{b) } U_1 \cup U_2 = \lbrace x = r\vecT{1\\0} \lor x = s\vecT{0\\1},\;r,\;s \in \R \rbrace \\
		      \colGreen{(UV0): \checkmark} &r = 0 \land s = 0\\
		      \colGreen{(UV2): \checkmark} &\\
		      \colGreen{(UV1): }\; \colRed{\lightning} \;&u = \vecT{1\\0},\quad v = \vecT{0\\1} \Rightarrow u + v = \vecT{1\\1} \notin U_1 \cup U_2
		    \end{align*}
		    \item[\textbf{5)}]
		    \begin{align*}
	        V &= \lbrace f: f:[-1,1] \rightarrow \R \rbrace \\
	        U &= \lbrace \usGreen{\text{wahrscheinlich g}}{f}: [-1,1] \rightarrow \R: f(0) = 0 \rbrace, \; UVR \text{ von } V?\\
	        \colGreen{(UV0): \checkmark} &da\; f(x) = 0 \forall x \in [-1,1] \in U\\
	        \colGreen{(UV1): \checkmark} &f,\;g \in U \Rightarrow f(0) = 0 \land g(0) = 0 \\
	        &(f+g)(0) = f(0)+g(0) = 0+0 = 0\\
	        &\Rightarrow f+g \in U\\
	        \colGreen{(UV2): \checkmark} &f \in U,\; \lambda \in \R \Rightarrow f(0) = 0 = (\lambda f)(0)\\
	        &= \lambda f(0) = \lambda 0 = 0 = r f \in U\\
	        \\&\text{\colRed{kontrollieren, irgendwie komisch}}	    
		    \end{align*}
		  \end{description}
		  
	  \subsubsection{Integralrechnung}
	  \subsubsubsection{$\int_1^2 \frac{ln(t)}{t} dt$}
		\begin{align}
		\colBord{s }\; &\colBord{= ln(t) \Rightarrow \frac{ds}{dt} \Rightarrow ds} \;\colBord{=} \colBord{\frac{1}{t}dt} \nonumber\\
		\colBord{Grenzen: t } &\colBord{=} \;\colBord{1, t=2 \rightarrow s = ln(1), s = ln(2)} \nonumber \\
		\int_1^2 \frac{ln(t)}{t} dt &= \int_1^2 ln(t) \frac{1}{t} dt = \int_{ln(1)}^{ln(2)} s ds 
		= \frac{1}{2} s^2 \colBord{\Big|_{s = ln(1) = 0}^{ln(2)}} = \frac{1}{2}(ln(2))^2 
		\end{align}
		Bauart des Integrals: $\int h(t)h'(t)dt$
		
		\subsubsubsection{$\int_0^{\frac{1}{2}} tan(t) dt$}
		\begin{align}
			\colBord{s = cos(t) \rightarrow \frac{ds}{dt}} \; &\colBord{=} \; \colBord{-sin(t) \rightarrow ds = -sin(t) dt} \nonumber \\
			\colBord{Grenzen: t = 0, t } \; &\colBord{=} \; \colBord{1 \rightarrow s = cos(0), s = cos(\frac{1}{2}} \nonumber \\
			\int_0^{\frac{1}{2}} tan(t) dt = \int_{0}{\frac{1}{2}} \frac{sin(t)}{cos(t)} dt &= \int_0 ^{\frac{1}{2}} \frac{1}{cos(t)} sin(t) dt 
			= - \int_{cos(0)}^{\frac{1}{2}} \frac{1}{s} ds \nonumber \\
			= -ln(s)\colBord{\Big|_{s = cos(0)}^{cos(\frac{1}{2})}} &= -ln(cos(\frac{1}{2})) + ln(\usGreen{=1}{cos(0)})
		\end{align}
		Bauart des Integrals: $\int \frac{1}{h(t)}h'(t)dt$
		
		\subsubsubsection{$\int 4xe^{-x} dx$}
		\begin{align}
			\int 4xe^{-x} dx &= 4 \int \usGreen{u}{x} \usGreen{v'}{e^{-x}} dx 
			\overset{\colGreen{p.I.}}{=} \usGreen{u}{x} \usGreen{v}{-e^{-x}} 
			-4\int \usGreen{u'}{1} \usGreen{v}{-e^{-x}} dx \nonumber\\
			&= -4xe^{-x} -4e^{-x} + C 
			\qquad C\in\R \label{eq:ex1}
		\end{align}
		
		\subsubsubsection{$\int_0^{\pi}(x+3)cos(2x) dx$)}
		\begin{align}
			\int_0^{\pi} \usGreen{u}{(x+3)} \usGreen{v'}{cos(2x)}dx 
			&\overset{\colGreen{p.I.}}{=} \underbrace{\usGreen{u}{(x+3)} * \frac{1}{2}\usGreen{v}{sin(2x)}\Big|_0^{\pi}}_{\colBord{=0}}
			- \frac{1}{2} \int_0^{\pi} \usGreen{u'}{1} * \usGreen{v}{sin(2x)} \nonumber \\
			&\;=\frac{1}{4}\; cos(2x)\Big|_0^{\pi} = 0 \label{eq:ex2}
		\end{align}
		
		\subsubsubsection{$\int cos^2(x)dx$}
		\begin{align}
			\int cos^2(x)\;dx &= \int \usGreen{u}{cos(x)} * \usGreen{v'}{cos(x)}\;dx \overset{\colGreen{p.I.}}{=}
			\usGreen{u}{cos(x)} * \usGreen{v}{sin(x)} - \int \usGreen{u'}{(-sin(x))}*\usGreen{v}{sin(x)}\;dx \nonumber\\
			&=cos(x)sin(x) + \int \underbrace{sin^2(x)}_{\colBord{= 1-cos^2(x)}}dx = cos(x)sin(x) + \int 1dx - \int cos^2(x)\; dx \nonumber\\
			&\Rightarrow 2\int cos^2(x)\;dx = cos(x)sin(x) + x + \tilde{c} \qquad ,\tilde{c}\in\R\nonumber\\
			&\Rightarrow \int cos^2(x)\;dx = \frac{cos(x)sin(x)+x}{2}+C \qquad, C\in\R \label{eq:ex3}
		\end{align}
		
		\subsubsubsection{$\int_0^1 x arctan(x) dx$}
		\begin{align}
			\int_0^1 \usGreen{u'}{x} \usGreen{v}{arctan(x)} dx &\overset{\colGreen{p.I.}}{=} 
			\frac{1}{2} * \usGreen{u}{x^2} \usGreen{v}{arctan(x)} \Big|_0^1 - 
			\int_0^1 \usGreen{u}{\frac{1}{2} x^2} \usGreen{v'}{\frac{1}{1+x^2}}dx \nonumber\\
			&\;= \frac{\pi}{8} - \frac{1}{2} \int_0^1 \frac{1}{1+x^2} dx 
			= \frac{\pi}{8} - \frac{1}{2} \int_0^1 1 dx + \int_0^1 \frac{1}{1+x^2} dx \nonumber\\
			&\;= \frac{\pi}{8} - \frac{1}{2}* (x\Big|_0^1) + \frac{1}{2} actan(x)\Big|_0^1 
			= \frac{\pi}{8} - \frac{1}{2} * \frac{\pi}{8} \nonumber\\
			&\;= \frac{\pi}{4} - \frac{1}{2} \label{eq:ex4}
		\end{align}
		
		\subsubsubsection{$\int \frac{3x^2+2}{x^3+2x+1}dx$}
		\begin{align}
		  \int \frac{3x^2+2}{x^3+2x+1}dx = ln(|x^2+2x-1|) + C \qquad ,C\in\R\label{eq:ex5}
		\end{align}
		Da das Integral über einen Ausdruck der Form $\frac{f'(x)}{f(x)}$ gebildet wird gilt \eqref{eq:rule_spec}. Alternativ Substitution:
		\begin{align*}
		  y = x^3+2x-1 \Rightarrow dy = (3x^2+2) dx \Rightarrow \int \frac{1}{y}dy
		\end{align*}
		
		\subsubsubsection{$\int cos(x) e^{3sin(x)}dx$}
		\begin{align}
			\colBord{y = 3sin(x) \Rightarrow dy}\; &\;\;\colBord{=} \; \colBord{3cos(x) dx} \nonumber\\
			\int cos(x) e^{3sin(x)}dx &\overset{\colGreen{subs.}}{=} \frac{1}{3} \int e^y dy \nonumber\\
			&\;= \frac{1}{3} e^y + C \usGreen{R.s.}{=} \frac{1}{3}e^{3sin(x)} + C \qquad , \; C\in\R
		\end{align}
		
		\subsubsubsection{$\int \frac{1}{(2+x) \sqrt{1+x}}dx$}
		\begin{align}
			\colBord{x = 2sin(y) \Rightarrow dx} &\;\;\colBord{=} \colBord{2cos(y) dy} \nonumber\\
			\int \frac{1}{(2+x) \sqrt{1+x}}dx &\usGreen{subs.}{=} 
			\int \frac{1}{(y^2 + 1) \colRed{\cancel{y}}} 2\colRed{\cancel{y}} dy 
			= 2\int \frac{1}{1+y^2}\;dy \nonumber\\
			&\;= 2*arctan(y) + C = 2* arctan(\sqrt{1+x} + C ,\;C\in\R
		\end{align}
		
		\subsubsubsection{$\int \sqrt{4+x^2} \; dx$}
		\begin{align}
			\colBord{x = 2 sin(y) \Rightarrow dx} &\;\;\colBord{=} \colBord{\;2cos(y)dy} \nonumber\\
			\int \sqrt{4+x^2} \; dx &\overset{\colGreen{subs.}}{=}
			\int \sqrt{4-4sin^2(y)} * 2cos(y) \; dy \nonumber\\
			&\;= \int \sqrt{4cos^2(y)} 2cos(y) \; dy = 4\int cos^2(y) \; dy \nonumber\\
			&\overset{\colGreen{p.I.}}{=} 4*(\usGreen{u}{cos(y)} * \usGreen{v}{sin(y)} - \int \usGreen{u'}{(-sin(y))}*\usGreen{v}{sin(y)}\;dy) \nonumber\\
			&\;\Rightarrow 4\int cos^2(y) dy = cos(y)sin(y) + \int 1dy - \int cos^2(y)\; dy \nonumber\\
			&\;\Rightarrow 5\int cos^2(y)\;dy = cos(y)sin(y) + y + \tilde{c} \qquad ,\tilde{c}\in\R\nonumber\\
			&\;\Rightarrow \int cos^2(y)\;dy = \frac{cos(y)sin(y)+x}{5}+C \qquad, C\in\R
		\end{align}
		
		\subsubsubsection{$\int_{-3}^3 1+e^{x^2} *(sin(x)^3\;dx$}
		\begin{align}
			\int_{-3}^3 1+e^{x^2} *(sin(x)^3\;dx &= \underbrace{\int_{-3}^3 1 dx}_{\colBord{= 6}} 
			+ \underbrace{\int_{-3}^{3} e^{x^2} * (sin(x))^2 dx}_{\colBord{ = 0 \text{ da der Integrand 
			eine ungerade Funktion ist}}} \nonumber\\
			&= 6
		\end{align}
		Untersucht man den Integranden, stellt man seine Ungeradheit leicht fest.
		\begin{align*}
			sin(x) &\Rightarrow ungerade \\
			sin^2(x) &\Rightarrow gerade \\
			sin^3(x) &\Rightarrow ungerade
		\end{align*} 
		Funktionen verhalten sich im Bezug auf Ungeradheit zur Multiplikation ähnlich wie Vorzeichen.
		($(-)*(-) = (+);\; (-) * (+) = (-)$)
		Untersucht man weiterhin die e-Funktion stellt man fest, dass sie gerade ist:
		\begin{align*}
		  f(x) = e^{x^2} = e^{(-x)^2} = f(-x) \Rightarrow gerade \; Funktion
		\end{align*}
		Da also eine gerade mit einer ungeraden Funktion multipliziert wird ist das Ergebnis wiederrum ungerade.  
	  
	  \subsubsection{Separierbare DGL}
	  \begin{description}
		    \item[\textbf{1)}]
		    \begin{align*}
		      y' &= t(1+y) \qquad,\;y(0) = 1\\
		      &\int \frac{1}{1+y} dy = \int t dt + C \inR{C}\\
		      &\Rightarrow ln |1+y| = \frac{1}{2}t^2 + C \Rightarrow |1+y| = e^{\frac{1}{2}t^2+C} \\
		      &\Rightarrow y = e^C e^{\frac{1}{2}t^2} -1 \Rightarrow y(t) = \ubGreen{\in \R^+}{d}e^{\frac{1}{2}t^2} -1
		      &\text{Anfangsbedingungen: } y(0) = 1 = d-1 \Rightarrow d = 2
		    \end{align*}
		    \item[\textbf{2)}]
		    \begin{align*}
			    y' &= e^{-y} \qquad ,\;y(2) = 0\\
			    &\int \frac{1}{e^{-y}} dy = \int 1dt + C \inR{C}\\
			    &\Rightarrow \int e^y dy = e^y = t+C\\
			    &\Rightarrow y(t) = ln(t+C), \quad y(2) = ln(2+C) \overset{!}{=} 0 \Rightarrow C = -1\\
			    &\Rightarrow y(t) = ln(t-1)
		    \end{align*}
		    \item[\textbf{3)}]
		    \begin{align*}
		      y' - y^2 &= 1 \qquad ,\;y(0) = 1\\
		      y' &= y^2 + 1 \Rightarrow \int \frac{1}{1+y^2}dy = \int 1dt +C \inR{C}\\
		      &\Rightarrow arctan(y) = t + C \Rightarrow y(t) = tan(t+C)\\
		      &y(0) = tan(C) \overset{!}{=} 1 \Rightarrow C = \frac{\pi}{4} \Rightarrow y(t) = tan(t+\frac{\pi}{4}) 
		    \end{align*}
		\end{description}
		
\newpage
\subsection{Formelverzeichnis}

\subsection{Nachwort}
Dieses Dokument versteht sich einzig als Zusammenfassung der Vorlesungsunterlagen aus der HM1-3 Vorlesung von Prof. Dr. Guido Schneider mit einigen zusätzlichen Beispielen. Der Sinn ist einzig mir selbst und meinen Kommilitonen das studieren der Mathematik zu erleichtern. In diesem Sinne erhebe ich keinerlei Anspruch auf das hier dargestellte Wissen, da es sich in großen Teilen nur um Neuformulierungen aus der Vorlesung und aus dem Begleitkurs vom Mint Kolleg handelt, in dem Frau Dr. Monika Schulz den Stoff bereits hervorragend zusammengefasst hat. Sollten sich einige Fehler eingeschlichen haben (was sehr wahrscheinlich ist) würde ich mich freuen, wenn man mir das kurz mitteilen würde damit ich eine Korrektur vornehmen kann. Das kann entweder über die Fachschaft erfolgen, oder gerne per E-Mail an f.leuze@outlook.de.

\bibliography{lit}

\end{document}